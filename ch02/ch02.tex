%%chapter%% 02
\chapter{To infinity --- and beyond!}\label{ch:infinity}

Little kids readily pick up the idea of infinity.\index{infinity}
``When I grow up, I'm
gonna have a million Barbies.'' ``Oh yeah? Well, I'm gonna have a billion.''
``Well, I'm gonna have infinity Barbies.'' ``So what? I'll have two infinity of them.''
Adults laugh, convinced that infinity, $\infty$, is the biggest number, so $2\infty$
can't be any bigger. This is the idea behind a joke in the movie Toy Story. Buzz
Lightyear's slogan is ``To infinity --- and beyond!'' We assume there \emph{isn't}
any beyond. Infinity is supposed to be the biggest there is, so by definition there
can't be anything bigger, right?

\section{Infinitesimals}\index{infinitesimal number}

\smallfig[t]{leibniz}{Gottfried Leibniz (1646-1716)}%
Actually mathematicians have invented many different logical systems for working
with infinity, and in most of them infinity does come in different sizes and flavors.
Newton, as well as the German mathematician Leibniz\index{Leibniz, Gottfried}
who invented calculus independently,\footnote{There is
some dispute over this point. Newton and his supporters claimed that Leibniz plagiarized Newton's
ideas, and merely invented a new notation for them.}
had a strong intuitive idea that calculus was really about numbers that were infinitely
small: infinitesimals, the opposite of infinities. For instance, consider
the number $1.1^2=1.21$. That 2 in the first decimal place is the same 2 that
appears in the expression $2t$ for the derivative of $t^2$.

%%graph%% close-up func=x**2 format=eps xlo=0.6 xhi=1.401 ylo=0.6 yhi=1.401 with=lines x=t y=x samples=300 xtic_spacing=.2 ytic_spacing=.2 border=1 ; func=2.1*x-1.1 with=lines
\smallfig{close-up}{A close-up view of the function $x=t^2$, showing the line that connects the points $(1,1)$ and $(1.1,1.21)$.}

Figure  \figref{close-up} shows the idea visually. The line connecting the points $(1,1)$ and $(1.1,1.21)$ is almost
indistinguishable from the tangent line on this scale. Its slope is $(1.21-1)/(1.1-1)=2.1$, which is very close to the
tangent line's slope of 2. It was a good approximation because the points were close together, separated by only
0.1 on the $t$ axis.

If we needed a better approximation, we could try calculating $1.01^2=1.0201$. The slope of the
line connecting the points $(1,1)$ and $(1.01,1.0201)$ is 2.01, which is even closer to the slope of the tangent line.

\smallfig[b]{t-squared-geometrically}{A geometrical interpretation of the derivative of $t^2$.}%
%
Another method of visualizing the idea is that we can interpret $x=t^2$ as the area of a square with sides of length $t$,
as suggested in figure \figref{t-squared-geometrically}. We increase $t$ by an infinitesimally small number $\der t$.
The $\der$ is Leibniz's notation for a very small difference,\index{Leibniz notation!infinitesimal}
and $\der t$ is to be read is a single symbol,
``dee-tee,'' not as a number $d$ multiplied by a number $t$. The idea is that $\der t$ is smaller than any ordinary
number you could imagine, but it's not zero. The area of the square is increased by $\der x = 2t\der t +\der t^2$, which is
analogous to the finite numbers $0.21$ and $0.0201$ we calculated earlier. Where before we divided by a finite
change in $t$ such as $0.1$ or $0.01$, now we divide by $\der t$, producing
\begin{align*}
  \frac{\der x}{\der t} &=  \frac{2t\:\der t +\der t^2}{\der t} \\
                        &= 2t+\der t
\end{align*}
for the derivative. On a graph like figure \figref{close-up}, $\der x/\der t$ is the slope of the tangent line: the change
in $x$ divided by the changed in $t$.\index{Leibniz notation!derivative}

But adding an infinitesimal number $\der t$ onto $2t$ doesn't really change it by any amount that's
even theoretically measurable in the real world, so the answer is really $2t$. Evaluating it at $t=1$ gives the exact result, 2,
that the earlier approximate results, 2.1 and 2.01, were getting closer and closer to.

\begin{eg}\label{eg:third-power}
To show the power of infinitesimals and the Leibniz notation, let's prove that the derivative of $t^3$ is $3t^2$:
\begin{align*}
  \frac{\der x}{\der t} &= \frac{(t+\der t)^3-t^3}{\der t} \\
                        &= \frac{3t^2\:\der t + 3t\:\der t^2 + \der t^3}{\der t} \\
                        &= 3t^2 + \ldots \qquad ,
\end{align*}
where the dots indicate infinitesimal terms that we can neglect.
\end{eg}

This result
required significant sweat and ingenuity when proved on page \pageref{detour:polynomial-proof} by the methods of chapter \ref{ch:rates-of-change},
and not only that but the old method would have required a completely different method of proof for a function that
wasn't a polynomial, whereas the new one can be applied more generally, as we'll see presently in examples
\ref{eg:derivative-of-one-over-x}-\ref{eg:geometric-series}.

It's easy to get the mistaken impression that infinitesimals exist in some remote fairyland where we can never
touch them. This may be true in the same artsy-fartsy sense that we can never truly understand $\sqrt{2}$, because
its decimal expansion goes on forever, and we therefore can never compute it exactly. But in practical work,
that doesn't stop us from working with $\sqrt{2}$. We just approximate it as, e.g., 1.41. Infinitesimals
are no more or less mysterious than irrational numbers, and in particular we can represent them concretely on a computer.
If you go to \verb@lightandmatter.com/calc/inf@, you'll find a web-based calculator called Inf,\index{Inf (calculator)}
which can handle infinite and infinitesimal numbers. It has a built-in symbol, \verb@d@, which represents an infinitesimally
small number such as the $\der x$'s and $\der t$'s we've been handling symbolically.

Let's use Inf to verify that the derivative of $t^3$, evaluated at $t=1$, is equal to 3, as found by plugging in to
the result of example \ref{eg:third-power}. The \verb@:@ symbol is the prompt that shows you Inf is ready to accept
your typed input.

\begin{Code}
  \ii : ((1+d)^3-1)/d
  \oo{3+3d+d^2}
\end{Code}

As claimed, the result is 3, or close enough to 3 that the infinitesimal error doesn't matter in real life.
It might look like Inf did this example by using algebra to simplify the expression, but in fact Inf doesn't
know anything about algebra. One way to see this is to use Inf to compare \verb@d@ with various real numbers:

\begin{Code}
  \ii : d<1
  \oo{true}
  \ii : d<0.01
  \oo{true}
  \ii : d<0.0000001
  \oo{true}
  \ii : d<0
  \oo{false}
\end{Code}

If \verb@d@ were just a variable being treated according to the axioms of algebra, there would be no way
to tell how it compared with other numbers without having some special information. Inf doesn't know
algebra, but it does know that \verb@d@ is a positive number that is less than any positive \emph{real} number
that can be represented using decimals or scientific notation.

\begin{eg}\label{eg:derivative-of-one-over-x}
In example \ref{eg:der-one-over-x-approx} on p.~\pageref{eg:der-one-over-x-approx}, we made a rough numerical
check to see if the differentiation rule $t^k \rightarrow kt^{k-1}$, which was proved on p.~\pageref{detour:polynomial-proof}
for $k=1,$ 2, 3, \ldots, was also valid for $k=-1$, i.e., for the function $x=1/t$. Let's look for an actual proof. To find a natural method of attack,
let's first redo the numerical check in a slightly more suggestive form. Again approximating the derivating at
$t=3$, we have
\begin{equation*}
  \frac{\der x}{\der t}  \approx \left(\frac{1}{3.01}-\frac{1}{3}\right)\left(\frac{1}{0.01}\right)  \qquad .
\end{equation*}
Let's apply the grade-school technique for subtracting fractions, in which we first get them over the same denominator:
\begin{equation*}
  \frac{1}{3}-\frac{1}{3.01} = \frac{3-3.01}{3\times3.01}  \qquad .
\end{equation*}
The result is 
\begin{align*}
  \frac{\der x}{\der t}  &\approx \left(\frac{-0.01}{3\times3.01}\right)\left(\frac{1}{0.01}\right) \\
                         &=-\frac{1}{3\times3.01} \qquad .
\end{align*}
Replacing 3 with $t$ and $0.01$ with $\der t$, this becomes
\begin{align*}
  \frac{\der x}{\der t}  &= -\frac{1}{t (t+\der t)} \\
                         &= -t^{-2}+\ldots
\end{align*}
\end{eg}

%%graph%% derivative-of-sine func=sin(x) format=eps xlo=0 xhi=6.283 ylo=-1 yhi=1 with=lines x=t y=x samples=300 xtic_spacing=1 ytic_spacing=1 ; func=cos(x)
\smallfig{derivative-of-sine}{Graphs of $\sin t$, and its derivative $\cos t$.}

\begin{eg}\label{eg:derivative-of-sin}\index{derivative!of the sine}\index{sine!derivative of}
The derivative of $x=\sin t$, with $t$ in units of radians, is
\begin{equation*}
  \frac{\der x}{\der t} = \frac{\sin(t+\der t)-\sin t}{\der t} \qquad , 
\end{equation*}
and with the trig identity $\sin(\alpha+\beta)=\sin\alpha\cos\beta+\cos\alpha\sin\beta$, this becomes
\begin{equation*}
                        = \frac{\sin t \: \cos \der t + \cos t \: \sin \der t - \sin t}{\der t} \qquad .
\end{equation*}
Applying the small-angle approximations $\sin u\approx u$ and $\cos u\approx 1$,
we have
\begin{align*}
\frac{\der x}{\der t}   &= \frac{\cos t \: \der t}{\der t} + \ldots\\
                        &= \cos t + \ldots \qquad ,
\end{align*}
where ``\ldots'' represents the error caused by the small-angle approximations.

This is essentially all there is to the computation of the derivative, except for the remaining technical
point that we haven't proved that the small-angle approximations are good enough. In example \ref{eg:third-power} on page \pageref{eg:third-power},
when we calculated the derivative of $t^3$, the resulting expression for the quotient $\der x/\der t$
came out in a form in which we could inspect the ``\ldots'' terms and verify before discarding them that they were infinitesimal.
The issue is less trivial in the present example.
This point is addressed more rigorously on page \pageref{detour:sin-rigor}.

Figure \figref{derivative-of-sine} shows the graphs of the function and its derivative. Note how the two
graphs correspond. At $t=0$, the slope of $\sin t$ is at its largest, and is positive; this is where
the derivative, $\cos t$, attains its maximum positive value of 1. At $t=\pi/2$, $\sin t$ has reached
a maximum, and has a slope of zero; $\cos t$ is zero here. At $t=\pi$, in the middle of the graph,
$\sin t$ has its maximum negative slope, and $\cos t$ is at its most negative extreme of $-1$.

Physically, $\sin t$ could represent the position of a pendulum as it moved back and forth from left
to right, and $\cos t$ would then be the pendulum's velocity.
\end{eg}

\begin{eg}\label{eg:dcos}\index{derivative!of the cosine}\index{cosine!derivative of}
What about the derivative of the cosine? The cosine and the sine are really the same function, shifted to the left or right by
$\pi/2$. If the derivative of the sine is the same as itself, but shifted to the left by $\pi/2$, then the derivative of
the cosine must be a cosine shifted to the left by $\pi/2$:
\begin{align*}
  \frac{\der\: \cos t}{\der t} &= \cos(t+\pi/2) \\
                               &= -\sin t \qquad .
\end{align*}
\end{eg}
The next example will require a little trickery.
By the end of this chapter you'll learn general techniques for
cranking out any derivative cookbook-style, without having to come
up with any tricks.

%%graph%% geometric-series func=1/(1-x) format=eps xlo=-1.0 xhi=1 ylo=0 yhi=3 with=lines x=t y=x samples=300 xtic_spacing=1 ytic_spacing=1
\smallfig{geometric-series}{The function $x=1/(1-t)$.}

\begin{eg}\label{eg:geometric-series}
\egquestion
Find the derivative of $1/(1-t)$, evaluated at $t=0$.

\eganswer
The graph shows what the function looks like. It blows up to infinity at $t=1$, but it's well behaved at $t=0$, where it has a
positive slope.

For insight, let's calculate some points on the curve. The point at which we're differentiating is
$(0,1)$. If we put in a small, positive value of $t$, we can observe how much the result increases
relative to 1, and this will give us an approximation to the derivative. For example, we find
that at $t=0.001$, the function has the value 1.001001001001, and so the derivative
is approximately $(1.001-1)/(.001-0)$, or about 1. We can therefore conjecture that the derivative
is exactly 1, but that's not the same as proving it.

But let's take another look at that number 1.001001001001. It's clearly a repeating decimal.
In other words, it appears that
\begin{equation*}
       \frac{1}{1-1/1000} = 1+\frac{1}{1000}+\left(\frac{1}{1000}\right)^2+\ldots \qquad ,
\end{equation*}
and we can easily verify this by multiplying both sides of the equation by $1-1/1000$ and collecting like
powers. This is a special case of the geometric series\index{geometric series}\index{series!geometric}\label{geometric-series}
\begin{equation*}
       \frac{1}{1-t} = 1+t+t^2+\ldots \qquad ,
\end{equation*}
which can be derived\footnote{As a technical aside, it's not necessary for our present purposes
to go into the issue of how to make the most general possible definition of what is meant by
a sum like this one which has an infinite number of terms; the only fact we'll need here is that
the error in \emph{finite} sum obtained by leaving out the ``\ldots'' has only higher powers of $t$. This
is taken up in more detail in ch.~\ref{ch:sequences}. Note that the series only gives the right answer
for $t<1$. E.g., for $t=1$, it equals $1+1+1+\ldots$, which, if it means anything, clearly means something
infinite.}
by doing synthetic division\index{synthetic division} (the equivalent of long
division for polynomials), or simply verified, after forming the conjecture based on the numerical
example above, by multiplying both sides by $1-t$.

As we'll see in section \ref{sec:safe-infinitesimals}, and have been implicitly assuming so far,
infinitesimals obey all the same elementary laws of algebra as the real numbers,
so the above derivation also holds for an infinitesimal value of $t$.
We can verify the result using Inf:
\begin{Code}
  \ii : 1/(1-d)
  \oo{1+d+d^2+d^3+d^4}
\end{Code}
Notice, however, that the series is truncated after the first five terms. This is similar to
the truncation that happens when you ask your calculator to find $\sqrt{2}$ as a decimal.

The result for the derivative is
\begin{align*}
  \frac{\der x}{\der t} &= \frac{\left(1+\der t+{\der t}^2+\ldots\right)-1}{1+\der t-1} \\
                        &= 1+\ldots \qquad .
\end{align*}
\end{eg}

%

\section{Safe use of infinitesimals}\label{sec:safe-infinitesimals}\index{infinitesimal number!safe use of}

The idea of infinitesimally small numbers has always irked purists. One prominent critic of the calculus was Newton's
contemporary George Berkeley, the Bishop of Cloyne. Although some of his complaints are clearly wrong (he denied the
possibility of the second derivative), there was clearly something to his criticism of the infinitesimals. He wrote
sarcastically,\index{infinitesimal number!criticism of}\index{Berkeley, George}
``They are neither finite quantities, nor quantities infinitely small, nor yet nothing. May we not call them ghosts of departed quantities?''

\smallfig[t]{berkeley}{Bishop George Berkeley (1685-1753)}%
%
Infinitesimals seemed scary, because if you mishandled them, you could prove absurd things. For example, let $\der u$ be
an infinitesimal. Then $2\der u$ is also infinitesimal. Therefore both $1/\der u$ and $1/(2\der u)$ equal infinity, so
$1/\der u = 1/(2\der u)$. Multiplying by $\der u$ on both sides, we have a proof that $1=1/2$.\label{bogus-proof}

In the eighteenth century, the use of infinitesimals became like adultery: commonly practiced, but shameful to admit
to in polite circles. Those who used them learned certain rules of thumb for handling them correctly. For instance,
they would identify the flaw in my proof of $1=1/2$ as my assumption that there was only one size of infinity,
when actually $1/\der u$ should be interpreted as an infinity twice as big as $1/(2\der u)$. The use of the symbol
$\infty$ played into this trap, because the use of a single symbol for infinity implied that infinities only came
in one size. However, the practitioners of infinitesimals had trouble articulating a clear set of principles
for their proper use, and couldn't prove that a self-consistent system could be built around them.

By the twentieth century, when I learned calculus, a clear consensus had formed that infinite and infinitesimal
numbers weren't numbers at all. A notation like $\der x/\der t$, my calculus teacher told me, wasn't really
one number divided by another, it was merely a symbol for something called a limit,\index{derivative!defined using a limit}\index{limit}
\begin{equation*}
\lim_{\Delta t\rightarrow 0} \frac{\Delta x}{\Delta t} \qquad ,
\end{equation*}
where $\Delta x$ and $\Delta t$ represented finite changes. I'll give a formal definition (actually two different formal
definitions) of the term ``limit'' in section \ref{sec:limits}, but intuitively the concept is that is that we can get as good
an approximation to the derivative as we like, provided that we make $\Delta t$ small enough.

That satisfied me until we got to a certain topic
(implicit differentiation) in which we were encouraged to break the $\der x$ away from the $\der t$, leaving them on
opposite sides of the equation. I buttonholed my teacher after class and asked why he was now doing what he'd
told me you couldn't really do, and his response was that $\der x$ and $\der t$ weren't really numbers,
but most of the time you could get away with treating them as if they were, and you would get the right
answer in the end. \emph{Most of the time!?} That bothered me. How was I supposed to know when it \emph{wasn't}
``most of the time?''

\smallfig{robinson}{Abraham Robinson (1918-1974)}

But unknown to me and my teacher, mathematician Abraham Robinson\index{Robinson, Abraham}
had already shown in the 1960's that it
was possible to construct a self-consistent number system that included infinite and infinitesimal numbers.
He called it the hyperreal number system,\index{hyperreal number}
and it included the real numbers as a subset.\footnote{The main text of this book treats infinitesimals
with the minimum fuss necessary in order to avoid the common goofs. More detailed
discussions are often relegated to the back of the book, as in example \ref{eg:derivative-of-sin} on page \pageref{eg:derivative-of-sin}.
The reader who
wants to learn even more about the hyperreal system should consult the list of further reading on page \pageref{ch:further-reading}.
}

Moreover, the
rules for what you can and can't do with the hyperreals turn out to be extremely simple. 
Take any true statement about the real numbers. Suppose it's possible to translate it into a statement about
the hyperreals in the most obvious way, simply by replacing the word ``real'' with the word ``hyperreal.''
Then the translated statement is also true. This is known as the \emph{transfer principle}.\index{transfer principle}

Let's look back at my bogus proof of $1=1/2$ in light of this simple principle. The final step of the proof,
for example, is perfectly valid: multiplying both sides of the equation by the same thing. The following
statement about the real numbers is true:

\begin{indentedblock}
For any real numbers $a$, $b$, and $c$, if $a=b$, then $ac=bc$.
\end{indentedblock}

This can be translated in an obvious way into a statement about the hyperreals:

\begin{indentedblock}
For any hyperreal numbers $a$, $b$, and $c$, if $a=b$, then $ac=bc$.
\end{indentedblock}

However, what about the statement that both $1/\der u$ and $1/(2\der u)$ equal infinity, so they're
equal to each other? This isn't the translation of a statement that's true about the reals, so there's
no reason to believe it's true when applied to the hyperreals --- and in fact it's false.

What the transfer principle tells us is that the real numbers as we normally
think of them are not unique in obeying the ordinary rules of algebra. There are completely different
systems of numbers, such as the hyperreals, that also obey them.

How, then, are the hyperreals even
different from the reals, if everything that's true of one is true of the other? But recall that
the transfer principle doesn't guarantee that every statement about the reals is also true of the
hyperreals. It only works if the statement about the reals can be translated into a statement
about the hyperreals in the most simple, straightforward way imaginable, simply by replacing
the word ``real'' with the word ``hyperreal.'' Here's an example of a true statement about the reals that
can't be translated in this way:

\begin{indentedblock}
For any real number $a$, there is an integer $n$ that is greater than $a$.
\end{indentedblock}

This one can't be translated so simplemindedly, because it refers to a subset of the reals called
the integers. It might be possible to translate it somehow, but it would require some insight into
the correct way to translate that word ``integer.'' The transfer principle doesn't apply to this
statement, which indeed is false for the hyperreals, because the hyperreals contain infinite
numbers that are greater than all the integers. In fact, the contradiction of this statement can be
taken as a definition of what makes the hyperreals special, and different from the reals: we assume
that there is at least one hyperreal number, $H$, which is greater than all the integers.

As an analogy from everyday life, consider the following statements about the student body of the
high school I attended:

\begin{indentedblock}
1. Every student at my high school had two eyes and a face.\\
2. Every student at my high school who was on the football team was a jerk.
\end{indentedblock}

Let's try to translate these into statements about the population of California in general.
The student body of my high school is like the set of real numbers, and the present-day population
of California is like the hyperreals. Statement 1 can be translated mindlessly into a statement
that every Californian has two eyes and a face; we simply substitute ``every Californian'' for
``every student at my high school.'' But statement 2 isn't so easy, because it refers to the
subset of students who were on the football team, and it's not obvious what the corresponding
subset of Californians would be. Would it include everybody who played high school, college,
or pro football? Maybe it shouldn't include the pros, because they belong to an organization
covering a region bigger than California. Statement 2 is the kind of statement that the
transfer principle doesn't apply to.\footnote{For a slightly more precise and formal statement
of the transfer principle, see page \pageref{transfer}.}\label{backref-transfer}

\begin{eg}\label{eg:halo}
As a nontrivial example of how to apply the transfer principle, let's consider how to handle
expressions like the one that occurred when we wanted to differentiate $t^2$ using infinitesimals:
\begin{equation*}
  \frac{\der \left(t^2\right)}{\der t} = 2t+\der t  \qquad .
\end{equation*}
I argued earlier than $2t+\der t$ is so close to $2t$ that for all practical purposes, the
answer is really $2t$. But is it really valid in general to say that $2t+\der t$ is the
same hyperreal number as $2t$? No. We can apply the transfer principle to the
following statement about the reals:

\begin{indentedblock}
For any real numbers $a$ and $b$, with $b\ne 0$, $a+b\ne a$.
\end{indentedblock}

Since $\der t$ isn't zero, $2t+\der t\ne 2t$.
\end{eg}

More generally, example \ref{eg:halo} leads us to visualize every number as being surrounded by
a ``halo''\index{halo}
of numbers that don't equal it, but differ from it by only an infinitesimal amount.
Just as a magnifying glass would allow you to see the fleas on a dog, you would need an infinitely
strong microscope to see this halo. This is similar to the idea that every integer is surrounded by a bunch of fractions that
would round off to that integer. We can define the \emph{standard part}\index{standard part} of a finite hyperreal
number, which means the unique real number that differs from it infinitesimally. For instance, the
standard part of $2t+\der t$, notated $\st(2t+\der t)$, equals $2t$. The derivative of a function
should actually be defined as the standard part of $\der x/\der t$, but we often write $\der x/\der t$
to mean the derivative, and don't worry about the distinction.\index{derivative!defined using infinitesimals}

One of the things Bishop Berkeley disliked about infinitesimals was the idea that
they existed in a kind of hierarchy, with $\der t^2$ being not just infinitesimally small, but infinitesimally
small compared to the infinitesimal $\der t$.
If $\der t$ is the flea on a dog, then $\der t^2$ is a submicroscopic
flea that lives on the flea, as in Swift's doggerel: ``Big fleas have little fleas/ On their backs to ride 'em,/
and little fleas have lesser fleas,/And so, ad infinitum.'' Berkeley's criticism was off the mark here: there is
such a hierarchy. Our basic assumption about the hyperreals was that they contain at least one infinite number,
$H$, which is bigger than all the integers. If this is true, then $1/H$ must be less than $1/2$, less than
$1/100$, less then $1/1,000,000$ --- less than $1/n$ for any integer $n$. Therefore the hyperreals are guaranteed
to include infinitesimals as well, and so we have at least three levels to the hierarchy: infinities comparable
to $H$, finite numbers, and infinitesimals comparable to $1/H$. If you can swallow that, then it's not too much
of a leap to add more rungs to the ladder, like extra-small infinitesimals that are comparable to $1/H^2$.
If this seems a little crazy, it may comfort you to think of statements about the hyperreals as descriptions
of limiting processes involving real numbers. For instance, in the sequence of numbers $1.1^2=1.21$,
$1.01^2=1.0201$, $1.001^2=1.002001$, \ldots, it's clear that the number represented by the digit 1 in the final decimal place is getting
smaller faster than the contribution due to the digit 2 in the middle.

\label{transcendentals}
One subtle issue here, which I avoided mentioning in the differentiation of the sine function on page \pageref{eg:derivative-of-sin},
is whether the transfer principle is sufficient to let us define all the functions that
appear as familiar keys on a calculator: $x^2$, $\sqrt{x}$, $\sin x$, $\cos x$, $e^x$, and so on.
After all, these functions were originally defined as rules that would take a real number as an input
and give a real number as an output. It's not trivially obvious that their definitions can naturally be extended
to take a hyperreal number as an input and give back a hyperreal as an output. Essentially the answer is that
we can apply the transfer principle to them just as we would to statements about simple arithmetic, but I've discussed
this a little more on page \pageref{detour:transcendentals}.

\section{The product rule}\index{derivative!product rule}\index{product rule}

When I first learned calculus, it seemed to me that if the derivative of $3t$ was $3$, and the derivative of
$7t$ was 7, then the derivative of $t$ multiplied by $t$ ought to be just plain old $t$, not $2t$. The reason there's
a factor of 2 in the correct answer is that $t^2$ has two reasons to grow as $t$ gets bigger: it grows because
the first factor of $t$ is increasing, but also because the second one is. In general, it's possible to find
the derivative of the product of two functions any time we know the derivatives of the individual functions.

\begin{important}[The product  rule]
If $x$ and $y$ are both functions of $t$, then the derivative of their product is
\begin{equation*}
  \frac{\der(xy)}{\der t} = \frac{\der x}{\der t}\cdot y +  x \cdot \frac{\der y}{\der t} \qquad .
\end{equation*}
\end{important}

The proof is easy. Changing $t$ by an infinitesimal amount $\der t$ changes the product $xy$ by
an amount
\begin{gather*}
  (x+\der x)(y+\der y)-xy \\
      = y\der x + x\der y+\der x\der y \qquad , \\
\intertext{and dividing by \der t makes this into}
      \frac{\der x}{\der t}\cdot y +  x \cdot \frac{\der y}{\der t} + \frac{\der x \der y}{\der t} \qquad ,
\end{gather*}
whose standard part is the result to be proved.

\begin{eg}
\egquestion Find the derivative of the function $t\sin t$.

\eganswer
\begin{align*}
  \frac{\der(t\sin t)}{\der t} &= t\cdot \frac{\der(\sin t)}{\der t}+\frac{\der t}{\der t}\cdot\sin t \\
           &= t\cos t+\sin t
\end{align*}
\end{eg}

Figure \figref{product-rule} gives the geometrical interpretation of the product rule. Imagine that the king, in his castle
at the southwest corner of his rectangular kingdom, sends out a line of infantry to expand his territory
to the north, and a line of cavalry to take over more land to the east. In a time interval $\der t$,
the cavalry, which moves faster, covers a distance $\der x$ greater than that covered by the infantry,
$\der y$. However, the strip of territory conquered by the cavalry, $y \der x$, isn't as great as it could have been,
because in our example $y$ isn't as big as $x$.
%
\fig{product-rule}{A geometrical interpretation of the product rule.}

A helpful feature of the Leibniz notation is that one can easily use it to check whether the units of an
answer make sense. If we measure distances in meters and time in seconds, then $xy$ has units of
square meters (area), and so does the change in the area, $\der(xy)$. Dividing by $\der t$ gives the
number of square meters per second being conquered. On the right-hand side of the product rule,
$\der x/\der t$ has units of meters per second (velocity), and multiplying it by
$y$ makes the units square meters per second, which is consistent with the left-hand side. The
units of the second term on the right likewise check out. Some beginners might be tempted to
guess that the product rule would be $\der(xy)/\der t=(\der x/\der t)(\der y/\der t)$, but the Leibniz
notation instantly reveals that this can't be the case, because then the units on the left,
$\munit^2/\sunit$, wouldn't match the ones on the right, $\munit^2/\sunit^2$.

Because this unit-checking feature is so helpful, there is a special way of writing a second
derivative in the Leibniz notation. What Newton called $\Ddot{x}$, Leibniz wrote as
\begin{equation*}
  \frac{\der^2 x}{\der t^2} \qquad .
\end{equation*}
Although the different placement of the 2's on top and bottom seems strange and inconsistent to many
beginners, it actually works out nicely.
If $x$ is a distance, measured in meters, and $t$ is a time, in units of seconds, then the
second derivative is supposed to have units of acceleration, in units of meters per second per
second, also written $(\munit/\sunit)/\sunit$, or $\munit/\sunit^2$. (The acceleration of falling
objects on Earth is $9.8\ \munit/\sunit^2$ in these units.) The Leibniz notation is meant to suggest
exactly this: the top of the fraction looks like it has units of meters, because we're not squaring $x$,
while the bottom of the fraction looks like it has units of seconds, because it looks like we're
squaring $\der t$. Therefore the units come out right. It's important to realize, however, that
the symbol $\der$ isn't a number (not a real one, and not a hyperreal one, either), so we can't really
square it; the notation is not to be taken as a literal statement about infinitesimals.

\begin{eg}\label{eg:derivative-of-sqrt}\index{derivative!of square root}
A tricky use of the product rule is to find the derivative of $\sqrt{t}$. Since $\sqrt{t}$ can be
written as $t^{1/2}$, we might suspect that the rule $\der(t^k)/\der t=kt^{k-1}$ would work,
giving a derivative $\frac{1}{2}t^{-1/2}=1/(2\sqrt{t})$. However, the method from ch.~\ref{ch:rates-of-change} used to prove
that rule proved on p.\pageref{detour:polynomial-proof} only work if $k$ is an integer, so the best we could do would be to confirm
our conjecture approximately by graphing or numerical estimation.

Using the product rule, we can write $f(t)=\der\sqrt{t}/\der t$ for our unknown derivative, and
back into the result using the product rule:
\begin{align*}
  \frac{\der t}{\der t} &= \frac{\der(\sqrt{t}\sqrt{t})}{\der t} \\
           &= f(t)\sqrt{t}+\sqrt{t}f(t) \\
           &= 2f(t)\sqrt{t}
\end{align*}
But $\der t/\der t=1$, so $f(t)=1/(2\sqrt{t})$ as claimed.
\end{eg}

The trick used in example \ref{eg:derivative-of-sqrt} can also be used to prove that the power
rule $\der(x^n)/\der x=nx^{n-1}$ applies to cases where $n$ is an integer less than 0, but
I'll instead prove this on page \pageref{eg:der-x-to-any-k} by a technique that doesn't depend on
a trick, and also applies to values of $n$ that aren't integers.\label{coy-der-x-to-neg-k}

\widefig[t]{clowns}{Three clowns on seesaws demonstrate the chain rule.}

\section{The chain rule}\label{sec:chain-rule}\index{derivative!chain rule}\index{chain rule}

Figure \figref{clowns} shows three clowns on seesaws. If the leftmost clown moves down by a distance $\der x$,
the middle one will come up by $\der y$, but this will also cause the one on the right to move down by $\der z$.
If we want to predict how much the rightmost clown will move in response to a certain amount of motion by the
leftmost one, we have
\begin{equation*}
  \frac{\der z}{\der x} =   \frac{\der z}{\der y}  \cdot \frac{\der y}{\der x} \qquad .
\end{equation*}
This is called the chain rule. It says that if a change in $x$ causes $y$ to change, and $y$ then causes $z$ to change,
then this chain of changes has a cascading effect. Mathematically, 
there is no big mystery here. We simply cancel $\der y$ on the top and bottom. The only minor subtlety
is that we would like to be able to be sloppy by using an expression like $\der y/\der x$ to mean
both the quotient of two infinitesimal numbers and a derivative, which is defined as the standard part
of this quotient. This sloppiness turns out to be all right, as proved on page \pageref{detour:chain-rule}.

\begin{eg}
\egquestion Jane hikes 3 kilometers in an hour, and hiking burns 70 calories per kilometer. At what rate
does she burn calories?

\eganswer We let $x$ be the number of hours she's spent hiking so far, $y$ the distance covered,
and $z$ the calories spent. Then
\begin{align*}
  \frac{\der z}{\der x} &= \left(\frac{70\ \zu{cal}}{1\ \cancel{\zu{km}}}\right)
                                     \left(\frac{3\ \cancel{\zu{km}}}{1\ \zu{hr}}\right) \\
          &= 210\ \zu{cal}/\zu{hr} \qquad .
\end{align*}
\end{eg}

\begin{eg}\label{eg:gear-ratio}
\egquestion Figure \ref{fig:gear-ratio} shows a piece of farm equipment containing a train of
gears with 13, 21, and 42 teeth. If the smallest gear is driven by a motor, relate the rate of
rotation of the biggest gear to the rate of rotation of the motor.

\eganswer
Let $x$, $y$, and $z$ be the angular positions of the three gears. Then by the chain rule,
\begin{align*}
  \frac{\der z}{\der x} &= \frac{\der z}{\der y}  \cdot \frac{\der y}{\der x} \\
                        &= \frac{13}{21} \cdot \frac{21}{42} \\
                        &= \frac{13}{42} \qquad .
\end{align*}

\end{eg}

\smallfig{gear-ratio}{Example \ref{eg:gear-ratio}.}


The chain rule lets us find the derivative of a function that has been built out of one function
stuck inside another.

\begin{eg}
\egquestion Find the derivative of the function $z(x)=\sin(x^2)$.

\eganswer Let $y(x)=x^2$, so that $z(x)=\sin(y(x))$. Then
\begin{align*}
  \frac{\der z}{\der x} &= \frac{\der z}{\der y} \cdot \frac{\der y}{\der x} \\
                        &= \cos(y) \cdot 2x \\
                        &= 2x \cos(x^2)
\end{align*}
The way people usually say it is that the chain rule tells you to take the derivative of
the outside function, the sine in this case, and then multiply by the derivative of
``the inside stuff,'' which here is the square. Once you get used to doing it, you don't
need to invent a third, intermediate variable, as we did here with $y$.
\end{eg}

\begin{eg}\label{eg:chain-rule-without-leibniz}
Let's express the chain rule without the use of the Leibniz notation.
Let the function $f$ be defined by $f(x)=g(h(x))$. Then the derivative
of $f$ is given by $f'(x)=g'(h(x))\cdot h'(x)$.
\end{eg}

\begin{eg}\label{eg:derive-neg-integer-power}
\egquestion We've already proved that the derivative of $t^k$ is $kt^{k-1}$ for $k=-1$
(example \ref{eg:derivative-of-one-over-x} on p.~\pageref{eg:derivative-of-one-over-x})
and for $k=1,$ 2, 3, \ldots (p.~\pageref{detour:polynomial-proof}). Use these facts to extend
the rule to all integer values of $k$.

\eganswer
For $k<0$, the function $x=t^k$ can be written as $x=(t^{-1})^{-k}$, where $-k$ is positive.
Applying the chain rule, we find $\der x/\der t=(-k)(t^{-1})^{-k-1}(-t^{-2})=kt^{k-1}$.
\end{eg}

\section{Exponentials and logarithms}

\subsection{The exponential}\index{derivative!of the exponential}\index{exponential!derivative of}
The exponential function $e^x$, where $e=2.71828\ldots$ is the base of natural
logarithms, comes constantly up in applications as diverse as credit-card interest, the growth of
animal populations, and electric circuits. For its derivative we have
\begin{align*}
  \frac{\der e^x}{\der x} &= \frac{e^{x+\der x}-e^x}{\der x} \\
                          &= \frac{e^x e^{\der x}-e^x}{\der x} \\
                          &= e^x \: \frac{e^{\der x}-1}{\der x}
\end{align*}
The second factor, $\left(e^{\der x}-1\right)/\der x$, doesn't have $x$ in it, so it
must just be a constant. Therefore we know that the derivative of $e^x$ is simply
$e^x$, multiplied by some unknown constant,
\begin{equation*}
  \frac{\der e^x}{\der x} = c\:e^x .
\end{equation*}

A rough check by graphing at, say $x=0$, shows that the slope is close to 1, so $c$ is close to
1. Numerical calculation also shows that, for example, $(e^{0.001}-1)/0.001=1.00050016670838$ is very close
to 1. But how do we know it's exactly one when $\der x$ is really infinitesimal?
We can use Inf:
\begin{Code}
  \ii : [exp(d)-1]/d
  \oo{1+0.5d+...}
\end{Code}
(The ... indicates where I've snipped some higher-order terms out of the output.) It seems clear
that $c$ is equal to 1 except for negligible terms involving higher powers of $\der x$.
A rigorous proof is given on page \pageref{detour:exp}.\label{main:exp}

\begin{eg}\label{eg:caffeine}
\egquestion The concentration of a foreign substance in the bloodstream generally falls off exponentially
with time as $c=c_{\zu{o}}e^{-t/a}$, where $c_{\zu{o}}$ is the initial concentration, and $a$ is a constant.
For caffeine in adults, $a$ is typically about 7 hours. An example is shown in figure \figref{caffeine}. Differentiate the concentration with respect
to time, and interpret the result. Check that the units of the result make sense.

\eganswer Using the chain rule,
\begin{align*}
  \frac{\der c}{\der t} &= c_{\zu{o}}e^{-t/a}\cdot\left(-\frac{1}{a}\right) \\
                        &= -\frac{c_{\zu{o}}}{a}e^{-t/a}
\end{align*}

This can be interpreted as the rate at which caffeine is being removed from the blood and put into the person's urine. It's
negative because the concentration is decreasing.
According to the original expression for $x$, a substance with a large $a$ will take a long time to reduce its concentration,
since $t/a$ won't be very big unless we have large $t$ on top to compensate for the large $a$ on the bottom.
In other words, larger values of $a$ represent substances that the body has a harder time getting rid of efficiently.
The derivative has $a$ on the bottom, and the interpretation of this is that for a drug that is hard to eliminate,
the rate at which it is removed from the blood is low.

It makes sense that $a$ has units of time, because the exponential function has to have a unitless argument, so the units
of $t/a$ have to cancel out. The units of the result come from the factor of $c_{\zu{o}}/a$, and
it makes sense that the units are concentration divided by time, because the result
represents the rate at which the concentration is changing.
\end{eg}
%%graph%% caffeine func=2*exp(-x/7) format=eps xlo=0 xhi=24 ylo=0 yhi=2.1 with=lines x=t y=c xtic_spacing=6 ytic_spacing=.5 more_space_below=5
\smallfig{caffeine}{Example \ref{eg:caffeine}. A typical graph of the concentration of caffeine in the blood, in units of milligrams per
liter, as a function of time, in hours.}

\begin{eg}
\egquestion Find the derivative of the function $y=10^x$.

\eganswer In general, one of the tricks to doing calculus is to rewrite functions in forms that
you know how to handle. This one can be rewritten as a base-$e$ exponent:
\begin{align*}
  y &= 10^x \\
  \ln y &= \ln\left(10^x\right) \\
  \ln y &= x \ln 10 \\
  y &= e^{x\ln 10}
\end{align*}
Applying the chain rule, we have the derivative of the exponential, which is just the same
exponential, multiplied by the derivative of the inside stuff:
\begin{align*}
  \frac{\der y}{\der x} &= e^{x\ln 10} \cdot \ln 10 \qquad .
\end{align*}
In other words, the ``$c$'' referred to in the discussion of the derivative of $e^x$ becomes
$c=\ln 10$ in the case of the base-10 exponential.
\end{eg}

\subsection{The logarithm}\index{derivative!of the logarithm}\index{logarithm!definition of}

The natural logarithm is the function that undoes the exponential. In a situation like this, we
have
\begin{equation*}
  \frac{\der y}{\der x} = \frac{1}{\der x/\der y} \qquad ,
\end{equation*}
where on the left we're thinking of $y$ as a function of $x$, and on the right we consider
$x$ to be a function of $y$. Applying this to the natural logarithm,
\begin{align*}
  y &= \ln x \\
  x &= e^y \\
  \frac{\der x}{\der y} &= e^y \\
  \frac{\der y}{\der x} &= \frac{1}{e^y} \\
                        & = \frac{1}{x} \\
  \frac{\der \ln x}{\der x} &= \frac{1}{x} \qquad .
\end{align*}

\fig{power-ladder}{Differentiation and integration of functions of the form $x^n$. Constants
out in front of the functions are not shown, so keep in mind that, for example, the derivative
of $x^2$ isn't $x$, it's $2x$.}

This is noteworthy because it shows that there must be an exception to the rule
that the derivative of $x^n$ is $nx^{n-1}$, and the integral of $x^{n-1}$ is
$x^n/n$. (On page \pageref{coy-der-x-to-neg-k} I remarked that this rule could
be proved using the product rule for negative integer values of $k$, but that I would
give a simpler, less tricky, and more general proof later. The proof is example
\ref{eg:der-x-to-any-k} below.) The integral of $x^{-1}$ is not $x^0/0$, which wouldn't make sense anyway
because it involves division by zero.\footnote{Speaking casually, one can say  that
division by zero gives infinity. This is often a good way to think when trying to
connect mathematics to reality. However, it doesn't really work that way according
to our rigorous treatment of the hyperreals. Consider this statement:
``For a nonzero real number $a$, there is no real number $b$ such that $a=0b$.''
This means that we can't divide $a$ by 0 and get $b$. Applying the transfer principle
to this statement, we see that the same is true for the hyperreals: division by zero
is undefined. However, we can divide a finite number by an infinitesimal, and get
an infinite result, which is almost the same thing.} Likewise the derivative of
$x^0=1$ is $0x^{-1}$, which is zero. Figure \figref{power-ladder} shows the idea.
The functions $x^n$ form a kind of ladder, with differentiation taking us down
one rung, and integration taking us up. However, there are two special cases where differentiation
takes us off the ladder entirely.

\begin{eg}\label{eg:der-x-to-any-k}
\egquestion Prove $\der(x^n)/\der x=nx^{n-1}$ for any real value of $n$, not just an integer.

\eganswer
\begin{align*}
  y &= x^n \\
    &= e^{n \ln x} \\
\intertext{By the chain rule,}
  \frac{\der y}{\der x} &= e^{n \ln x} \cdot \frac{n}{x} \\
                        &= x^n \cdot \frac{n}{x} \\
                        &= nx^{n-1} \qquad .
\end{align*}
(For $n=0$, the result is zero.)
\end{eg}

When I started the discussion of the derivative of the logarithm, I wrote $y=\ln x$ right
off the bat. That meant I was implicitly assuming $x$ was positive.\label{log-neg}
More generally, the derivative of $\ln|x|$ equals $1/x$, regardless of the sign (see
problem \ref{hw:log-neg} on page \pageref{hw:log-neg}).

\section{Quotients}\index{derivative!of a quotient}\index{quotient!derivative of}

So far we've been successful with a divide-and-conquer approach to differentiation:
the product rule and the chain rule offer methods of breaking a function down into
simpler parts, and finding the derivative of the whole thing based on knowledge of
the derivatives of the parts. We know how to find the derivatives of sums, differences,
and products, so the obvious next step is to look for a way of handling division.
This is straightforward, since we know that the derivative of the function $1/u=u^{-1}$
is $-u^{-2}$.
Let $u$ and $v$ be functions of $x$. Then by the product rule,
\begin{align*}
  \frac{\der(v/u)}{\der x} &= \frac{\der v}{\der x}\cdot\frac{1}{u} + v\cdot \frac{\der(1/u)}{\der x} \\
\intertext{and by the chain rule,}
  \frac{\der(v/u)}{\der x} &= \frac{\der v}{\der x}\cdot\frac{1}{u} - v\cdot \frac{1}{u^2}\frac{\der u}{\der x}
\end{align*}
This is so easy to rederive on demand that I suggest not memorizing it.

By the way, notice how the notation becomes a little awkward when we want to write a derivative like
$\der(v/u)/\der x$. When we're differentiating a complicated function, it can be uncomfortable
trying to cram the expression into the top of the $\der\ldots/\der\ldots$ fraction. Therefore
it would be more common to write such an expression like this:
\begin{equation*}
  \frac{\der}{\der x} \left(\frac{v}{u}\right)
\end{equation*}
This could be considered an abuse of notation, making $\der$ look like a number being
divided by another number $\der x$, when actually $\der$ is meaningless on its own. On the other hand,
we can consider the symbol $\der/\der x$ to represent the operation of differentiation with respect to
$x$; such an interpretation will seem more natural to those who have been inculcated with the taboo
against considering infinitesimals as numbers in the first place.

Using the new notation, the quotient rule becomes
\begin{equation*}
  \frac{\der}{\der x} \left(\frac{v}{u}\right) = \frac{1}{u}\cdot\frac{\der v}{\der x} - \frac{v}{u^2} \cdot \frac{\der u}{\der x} \qquad .
\end{equation*}
The interpretation of the minus sign is that if $u$ increases, $v/u$ decreases.

\begin{eg}
\egquestion Differentiate $y=x/(1+3x)$, and check that the result makes sense.

\eganswer We identify $v$ with $x$ and $u$ with $1+x$. The result is
\begin{align*}
  \frac{\der}{\der x} \left(\frac{v}{u}\right) &= \frac{1}{u}\cdot\frac{\der v}{\der x} - \frac{v}{u^2} \cdot \frac{\der u}{\der x} \\
            &= \frac{1}{1+3x} - \frac{3x}{(1+3x)^2}
\end{align*}
One way to check that the result makes sense it to consider extreme values of $x$. For very large values of $x$, the
1 on the bottom of $x/(1+3x)$ becomes negligible compared to the $3x$, and the function $y$ approaches $x/3x=1/3$ as a limit.
Therefore we expect that the derivative $\der y/\der x$ should approach zero, since the derivative of a constant is
zero. It works: plugging in bigger and bigger numbers for $x$ in the expression for the derivative does give
smaller and smaller results. (In the second term, the denominator gets bigger faster than the numerator, because
it has a square in it.)

Another way to check the result is to verify that the units work out. Suppose arbitrarily that $x$ has units of gallons.
(If the 3 on the bottom is unitless, then the 1 would have to represent 1 gallon, since you can't add things that have
different units.) The function $y$ is defined by an expression with units of gallons divided by gallons, so $y$ is
unitless. Therefore the derivative $\der y/\der x$ should have units of inverse gallons. Both terms in the
expression for the derivative do have those units, so the units of the answer check out.
\end{eg}

\section{Differentiation on a computer}\index{differentiation!computer-aided}
In this chapter you've learned a set of rules for evaluating derivatives: derivatives of products,
quotients, functions inside other functions, etc. Because these rules exist, it's always
possible to find a formula for a function's derivative, given the formula for the original
function. Not only that, but there is no real creativity required, so a computer can be
programmed to do all the drudgery. For example, you can download a free, open-source program
called Yacas from \verb@yacas.sourceforge.net@
%
%\noindent \verb@yacas.sourceforge.net@
%
and install it on a Windows or Linux machine. There is even a version you can run in a web
browser without installing any special software:
\verb@http://yacas.sourceforge.net/@
\verb@yacasconsole.html  .@\\
 A typical session with Yacas looks like this:\index{differentiation!computer-aided!symbolic}

\restartLineNumbers
\begin{eg}
\startcodeeg
\begin{Code}
  \ii D(x) x^2
  \oo{2*x}
  \ii D(x) Exp(x^2)
  \oo{2*x*Exp(x^2)}
  \ii D(x) Sin(Cos(Sin(x)))
  \oo{-Cos(x)*Sin(Sin(x))}
  \oo{   *Cos(Cos(Sin(x)))}
\end{Code}
\finishcodeeg
\end{eg}
Upright type represents your input, and italicized type is
the program's output.

First I asked it to differentiate $x^2$ with respect to x, and it told me the result was $2x$.
Then I did the derivative of $e^{x^2}$, which
I also could have done fairly easily by hand.
(If you're trying this out on a computer as you real along, make sure to capitalize functions
like Exp, Sin, and Cos.)
Finally I tried an example where I didn't know the answer off the top of my head, and that would
have been a little tedious to calculate by hand.

Unfortunately things are a little less rosy in the world of integrals. There are a few rules
that can help you do integrals, e.g., that the integral of a sum equals the sum of the
integrals, but the rules don't cover all the possible cases. Using Yacas to evaluate the
integrals of the same functions, here's what happens.\footnote{If you're trying these on your own
computer, note that the long input line for the function $\sin\:\cos\:\sin x$ shouldn't
be broken up into two lines as shown in the listing.}\index{integration!computer-aided!symbolic}

\restartLineNumbers
\begin{eg}
\startcodeeg
\begin{Code}
  \ii Integrate(x) x^2
  \oo{x^3/3}
  \ii Integrate(x) Exp(x^2)
  \oo{Integrate(x)Exp(x^2)}
  \ii{Integrate(x) }
  \cc{Sin(Cos(Sin(x)))}
  \oo{Integrate(x)}
  \oo{    Sin(Cos(Sin(x)))}
\end{Code}
\finishcodeeg
\end{eg}

The first one works fine, and I can easily verify that the answer is correct, by
taking the derivative of $x^3/3$, which is $x^2$. (The answer could have been $x^3/3+7$,
or $x^3/3+c$, where $c$ was any constant, but Yacas doesn't bother to tell us that.)
The second and third ones don't
work, however; Yacas just spits back the input at us without making any progress on it.
And it may not be because Yacas isn't smart enough to figure out these integrals.
The function $e^{x^2}$ can't be integrated at all in terms of a formula containing
ordinary operations and functions such as addition, multiplication, exponentiation,
trig functions, exponentials, and so on.

That's not to say that a program like this is useless. For example, here's an integral
that I wouldn't have known how to do, but that Yacas handles easily:

\restartLineNumbers
\begin{eg}
\startcodeeg
\begin{Code}
  \ii Integrate(x) Sin(Ln(x))
  \oo{(x*Sin(Ln(x)))/2}
  \oo{    -(x*Cos(Ln(x)))/2}
\end{Code}
\finishcodeeg
\end{eg}

This one is easy to check by differentiating, but I could have been marooned on a
desert island for a decade before I could have figured it out in the first place.
There are various rules, then, for integration, but they don't cover all possible
cases as the rules for differentiation do, and sometimes it isn't obvious
which rule to apply. Yacas's ability to integrate $\sin\:\ln x$ shows that it
had a rule in its bag of tricks that I don't know, or didn't remember, or didn't realize
applied to this integral.

Back in the 17th century, when Newton and Leibniz invented calculus, there were no
computers, so it was a big deal to be able to find a simple formula for your result.
Nowadays, however, it may not be such a big deal. Suppose I want to find the derivative
of $\sin\:\cos\:\sin x$, evaluated at $x=1$. I can do something like this on a calculator:

\restartLineNumbers
\begin{eg}
\startcodeeg
\begin{Code}
  \ii sin cos sin 1 =  
  \oo{0.61813407}
  \ii sin cos sin 1.0001 =
  \oo{0.61810240}
  \ii(0.61810240-0.61813407) 
  \cc{/.0001 = }
  \oo{-0.3167}
\end{Code}
\finishcodeeg
\end{eg}

I have the right answer, with plenty of precision for most realistic applications,
although I might have never guessed that the mysterious number $-0.3167$ was
actually $-(\cos 1)(\sin\sin 1)(\cos\cos\sin 1)$. This could get a little tedious if I wanted to graph the function,
for instance, but then I could just use a computer spreadsheet, or write a little
computer program. In this chapter, I'm going to show you how to do derivatives
and integrals using simple computer programs, using Yacas.
The following little Yacas program does the same thing as the set of calculator
operations shown above:\index{differentiation!computer-aided!numerical}

\pagebreak

\restartLineNumbers
\begin{eg}
\startcodeeg
\begin{Code}
  \nn f(x):=Sin(Cos(Sin(x)))
  \nn x:=1
  \nn dx:=.0001
  \nn N( (f(x+dx)-f(x))/dx )
  \oo{-0.3166671628}
\end{Code}
\finishcodeeg
\end{eg}


(I've omitted all of Yacas's output except for the final result.)
Line 1 defines the function we want to differentiate. Lines 2 and
3 give values to the variables x and dx. Line 4 computes the derivative;
the \verb@N( )@ surrounding the whole thing is our way of telling Yacas
that we want an approximate numerical result, rather than an exact symbolic one.

An interesting thing to try now is to make dx smaller and smaller, and see if
we get better and better accuracy in our approximation to the derivative.

\begin{eg}\label{eg:derivative-limit}
\startcodeeg
\begin{Code}
  \nn g(x,dx):= 
  \cc N( (f(x+dx)-f(x))/dx )
  \nn g(x,.1)
  \oo{-0.3022356406}
  \nn g(x,.0001)
  \oo{-0.3166671628}
  \nn g(x,.0000001)
  \oo{-0.3160458019}
  \nn g(x,.00000000000000001)
  \oo{0}
\end{Code}
\end{eg}

Line 5 defines the derivative function. It needs to know both x and dx. Line
6 computes the derivative using $\der x=0.1$, which we expect to be a lousy approximation,
since $\der x$ is really supposed to be infinitesimal, and $0.1$ isn't even that small.
Line 7 does it with the same value of $\der x$ we used earlier. The two results agree
exactly in the first decimal place, and approximately in the second, so we can be
pretty sure that the derivative is $-0.32$ to two figures of precision. Line 8
ups the ante, and produces a result that looks accurate to at least 3 decimal places.
Line 9 attempts to produce fantastic precision by using an extremely small value of $\der x$.
Oops --- the result isn't better, it's worse! What's happened here is that Yacas
computed $f(x)$ and $f(x+\der x)$, but they were the same to within the precision it
was using, so $f(x+\der x)-f(x)$ rounded off to zero.\footnote{Yacas can do arithmetic
to any precision you like, although you may run into practical limits due to the amount
of memory your computer has and the speed of its CPU. For fun, try \verb@N(Pi,1000)@,
which tells Yacas to compute $\pi$ numerically to 1000 decimal places.}

Example \ref{eg:derivative-limit} demonstrates the concept of how a derivative can be
defined in terms of a limit:\index{derivative!defined using a limit}
\begin{equation*}
  \frac{\der y}{\der x} = \lim_{\Delta x\rightarrow 0} \frac{\Delta y}{\Delta x}
\end{equation*}
The idea of the limit is that we can theoretically make $\Delta y/\Delta x$ approach
as close as we like to $\der y/\der x$, provided we make $\Delta x$ sufficiently small.
In reality, of course, we eventually run into the limits of our ability to do the
computation, as in the bogus result generated on line 9 of the example.





\begin{hwsection}

\begin{hwwithsoln}{fourth-power}
Carry out a calculation like the one in example \ref{eg:third-power} on page \pageref{eg:third-power} to
show that the derivative of $t^4$ equals $4t^3$.
\end{hwwithsoln}

\begin{hwwithsoln}{derivative-of-cos}
Example \ref{eg:dcos} on page \pageref{eg:dcos} gave a tricky argument to show that the derivative of
$\cos t$ is $-\sin t$. Prove the same result using the method of example \ref{eg:derivative-of-sin} instead.
\end{hwwithsoln}

\begin{hwwithsoln}{infinite-subtraction}
Suppose $H$ is a big number. Experiment on a calculator to figure out whether $\sqrt{H+1}-\sqrt{H-1}$
comes out big, normal, or tiny. Try making $H$ bigger and bigger, and see if you observe a trend.
Based on these numerical examples, form a conjecture about what happens to this expression when $H$ is infinite.
\end{hwwithsoln}

\begin{hwwithsoln}{infinitesimal-sqrt}
Suppose $\der x$ is a small but finite number. Experiment on a calculator to figure out how $\sqrt{\der x}$
compares in size to $\der x$.  Try making $\der x$ smaller and smaller, and see if you observe a trend.
Based on these numerical examples, form a conjecture about what happens to this expression when $\der x$ is
infinitesimal.
\end{hwwithsoln}

\begin{hwwithsoln}{transfer}
To which of the following statements can the transfer principle be applied? If you think it can't be applied
to a certain statement, try to prove that the statement is false for the hyperreals, e.g., by giving a
counterexample.

(a) For any real numbers $x$ and $y$, $x+y=y+x$.\\
(b) The sine of any real number is between $-1$ and 1.\\
(c) For any real number $x$, there exists another real number $y$ that is greater than $x$.\\
(d) For any real numbers $x\ne y$, there exists another real number $z$ such that $x<z<y$.\\
(e) For any real numbers $x\ne y$, there exists a rational number $z$ such that $x<z<y$. (A rational number is
one that can be expressed as an integer divided by another integer.)\\
(f) For any real numbers $x$, $y$, and $z$, $(x+y)+z=x+(y+z)$.\\
(g) For any real numbers $x$ and $y$, either $x<y$ or $x=y$ or $x>y$.\\
(h) For any real number $x$, $x+1\ne x$.
\end{hwwithsoln}

\begin{hwwithsoln}{parallel-resistance}
If we want to pump air or water through a pipe, common sense tells us that it will be easier
to move a larger quantity more quickly through a fatter pipe. Quantitatively, we can define
the resistance, $R$, which is the ratio of the pressure difference produced by the pump to the
rate of flow. A fatter pipe will have a lower resistance. Two pipes can be used in parallel,
for instance when you turn on the water both in the kitchen and in the bathroom, and in this
situation, the two pipes let more water flow than either would have let flow by itself, which
tells us that they act like a single pipe with some lower resistance. The equation for their
combined resistance is $R=1/(1/R_1+1/R_2)$. Analyze the case where one resistance is finite,
and the other infinite, and give a physical interpretation. Likewise, discuss the case where
one is finite, but the other is infinitesimal.
\end{hwwithsoln}

\begin{hwwithsoln}{velocity-addition-infinitesimals}
Naively, we would imagine that if a spaceship traveling at $u=3/4$ of the speed of light was to shoot a missile
in the forward direction at $v=3/4$ of the speed of light (relative to the ship), then the missile would be traveling at $u+v=3/2$ of the
speed of light. However, Einstein's theory of relativity tells us that this is too good to be true, because
nothing can go faster than light. In fact, the relativistic equation for combining velocities in this way
is not $u+v$, but rather $(u+v)/(1+uv)$. In ordinary, everyday life, we never travel at speeds anywhere near
the speed of light. Show that the nonrelativistic result is recovered in the case where both $u$ and $v$ are
infinitesimal.
\end{hwwithsoln}

\begin{hwwithsoln}{hundredth}
Differentiate  $(2x+3)^{100}$ with respect to $x$.
\end{hwwithsoln}

\begin{hwwithsoln}{one-and-two-hundred}
Differentiate  $(x+1)^{100}(x+2)^{200}$ with respect to $x$.
\end{hwwithsoln}

\begin{hwwithsoln}{ex-chain}
Differentiate  the following with respect to $x$: $e^{7x}$, $e^{e^x}$. (In the latter expression,
as in all exponentials nested inside exponentials, the evaluation proceeds from the top down, i.e.,
$e^{(e^x)}$, not $(e^e)^x$.)
\end{hwwithsoln}

\begin{hwwithsoln}{sinusoidal}
Differentiate $a\sin(bx+c)$ with respect to $x$.
\end{hwwithsoln}

\begin{hwwithsoln}{derivative-of-fractional-exp}
Let $x=t^{p/q}$, where $p$ and $q$ are positive integers. By a technique similar to the one
in example \ref{eg:derive-neg-integer-power} on p.~\pageref{eg:derive-neg-integer-power},
prove that the differentiation rule for $t^k$ holds when $k=p/q$.qwe
\end{hwwithsoln}

\begin{hwwithsoln}{integrate-sinusoidal}
Find a function whose derivative with respect to $x$ equals $a\sin(bx+c)$. That is, find an
integral of the given function.
\end{hwwithsoln}

\begin{hwwithsoln}{square-three-times}
Use the chain rule to differentiate $((x^2)^2)^2$, and show that you get the
same result you would have obtained by differentiating $x^8$.
\givecredit{M. Livshits}
\end{hwwithsoln}


\begin{hwwithsoln}{max-range}
The range of a gun, when elevated to an angle $\theta$, is given by
\begin{align*}
  R=\frac{2v^2}{g}\sin\theta\:\cos\theta \qquad .
\end{align*}
Find the angle that will produce the maximum range.
\end{hwwithsoln}

\begin{hwwithsoln}{cosh}\index{hyperbolic cosine}
The hyperbolic cosine function is defined by
\begin{align*}
  \cosh x = \frac{e^x+e^{-x}}{2} \qquad .
\end{align*}
Find any minima and maxima of this function.
\end{hwwithsoln}

\begin{hwwithsoln}{sin-sin-sin}
Show that the function $\sin(\sin(\sin x))$ has maxima and minima at all the same places
where $\sin x$ does, and at no other places.
\end{hwwithsoln}

\begin{hwwithsoln}{absval-derivative}
Let $f(x)=|x|+x$ and $g(x)=x|x|+x$. Find the derivatives of these functions at $x=0$ in terms of
(a) slopes of tangent lines and (b) infinitesimals.
\end{hwwithsoln}

\begin{hwwithsoln}{air-res-v}
In free fall, the acceleration will not be exactly constant, due to air resistance. For example,
a skydiver does not speed up indefinitely until opening her chute, but rather approaches a certain
maximum velocity at which the upward force of air resistance cancels out the force of gravity.
The expression for the distance dropped by of a free-falling object, with air resistance, is\footnote{Jan Benacka
and Igor Stubna, \emph{The Physics Teacher}, 43 (2005) 432.}
\begin{equation*}
  d = A \ln\left[\cosh\left(t\sqrt{\frac{g}{A}}\right)\right] \qquad ,
\end{equation*}
where $g$ is the acceleration the object would have without air resistance, the function cosh
has been defined in problem \ref{hw:cosh}, and
$A$ is a constant that depends on the size, shape, and mass of the object, and the density of
the air. (For a sphere of mass $m$ and diameter $d$ dropping in air, $A=4.11m/d^2$.
Cf. problem \ref{hw:air-res-taylor}, p. \pageref{hw:air-res-taylor}.)\\
(a) Differentiate this expression to find the velocity.  Hint: In order to simplify the writing,
start by defining some other symbol to stand for the constant $\sqrt{g/A}$.\\
(b) Show that your answer can be reexpressed in terms of the function tanh defined by
$\tanh x=(e^x-e^{-x})/(e^x+e^{-x})$.\index{hyperbolic tangent}\\
(c) Show that your result for the velocity approaches
a constant for large values of $t$.\\
(d) Check that your answers to parts b and c have units of velocity.
\end{hwwithsoln}

\begin{hwwithsoln}{derivative-of-tan}
Differentiate $\tan\theta$ with respect to $\theta$.
\end{hwwithsoln}

\begin{hwwithsoln}{cube-root}
Differentiate $\sqrt[3]{x}$ with respect to $x$.
\end{hwwithsoln}

\begin{hwwithsoln}{thompson-sqrts}
Differentiate the following with respect to $x$:\\
(a) $y=\sqrt{x^2+1}$ \\
(b) $y=\sqrt{x^2+a^2}$ \\
(c) $y=1/\sqrt{a+x}$ \\
(d) $y=a/\sqrt{a-x^2}$ \\
 \thompson
\end{hwwithsoln}

\begin{hwwithsoln}{logarithmy}
Differentiate $\ln(2t+1)$ with respect to $t$.
\end{hwwithsoln}

\begin{hwwithsoln}{unnecessary-prod}
If you know the derivative of $\sin x$, it's not necessary to use the product rule in order
to differentiate $3\sin x$, but show that using the product rule gives the right result anyway.
\end{hwwithsoln}

\begin{hwwithsoln}{gamma}
The $\Gamma$ function (capital Greek letter gamma) is a continuous mathematical function that
has the property $\Gamma(n)=1\cdot2\cdot\ldots\cdot(n-1)$ for $n$ an integer. $\Gamma(x)$ is also well
defined for values of $x$ that are not integers, e.g., $\Gamma(1/2)$ happens to be $\sqrt{\pi}$.
Use computer software that is capable of evaluating the $\Gamma$ function to determine numerically
the derivative of $\Gamma(x)$ with respect to $x$, at $x=2$. (In Yacas, the function is called Gamma.)
\end{hwwithsoln}

\begin{hwwithsoln}{cylinder}
For a cylinder of fixed surface area, what proportion of length to radius will give the maximum volume?
\end{hwwithsoln}

\begin{hwwithsoln}{relativistic-ke}
This problem is a variation on problem \ref{hw:ke} on page \pageref{hw:ke}. Einstein found that the
equation $K=(1/2)mv^2$ for kinetic energy was only a good approximation for speeds much less than
the speed of light, $c$. At speeds comparable to the speed of light, the correct equation is
\begin{equation*}
  K = \frac{\frac{1}{2}mv^2}{\sqrt{1-v^2/c^2}} \qquad .
\end{equation*}
(a) As in the earlier, simpler problem, find the power $\der K/\der t$ for an object accelerating
at a steady rate, with $v=at$. \\
(b) Check that your answer has the right units.\\
(c) Verify that the power required becomes infinite in the limit as $v$ approaches $c$, the speed of
light. This means that no material object can go as fast as the speed of light.
\end{hwwithsoln}

\begin{hwwithsoln}{log-neg}
Prove, as claimed on page \pageref{log-neg},
that the derivative of $\ln |x|$ equals $1/x$, for both positive and negative $x$.
\end{hwwithsoln}

\begin{hwwithsoln}{odd-even}
On even function is one with the property $f(-x)=f(x)$. For example, $\cos x$ is an even function, and $x^n$ is an even function if
$n$ is even. An odd function has $f(-x)=-f(x)$. Prove that the derivative of an even function is odd.
\end{hwwithsoln}

\begin{hw}
Suppose we have a list of numbers $x_1,\ldots x_n$, and we wish to find some number $q$ that is as close as possible
to as many of the $x_i$ as possible. To make this a mathematically precise goal, we need to define some numerical
measure of this closeness. Suppose we let $h=(x_1-q)^2+\ldots+(x_n-q)^2$, which can also be notated using $\Sigma$, uppercase Greek
sigma, as $h=\sum_{i=1}^n (x_i-q)^2$. Then minimizing $h$ can be used as a definition of optimal closeness.
(Why would we not want to use $h=\sum_{i=1}^n (x_i-q)$?) Prove that the value of $q$ that minimizes $h$ is
the average of the $x_i$.
\end{hw}

\begin{hwwithsoln}[2]{neg-power-trick}
Use a trick similar to the one used in example \ref{eg:derivative-of-sqrt} to prove that the power
rule $\der(x^k)/\der x=kx^{k-1}$ applies to cases where $k$ is an integer less than 0.
\end{hwwithsoln}

\begin{hwwithsoln}[2]{nonstandard-plane}
The plane of Euclidean geometry is today often described as the set of all coordinate pairs $(x,y)$, where
$x$ and $y$ are real. We could instead imagine the plane F that is defined in the same way, but with $x$ and $y$ taken from the
set of hyperreal numbers. As a third alternative, there is the plane G in which the finite hyperreals are used.
In E, Euclid's parallel postulate holds: given a line and a point not on the line, there exists exactly one line passing
through the point that does not intersect the line. Does the parallel postulate hold in F? In G?
Is it valid to associate only E with the plane described by Euclid's axioms?
\end{hwwithsoln}

\begin{hwwithsoln}{nines-forever}
Discuss the following statement: \emph{The repeating decimal $0.999\ldots$ is infinitesimally less than one}.
\end{hwwithsoln}

\begin{hwwithsoln}{chain-rule-units}
Example \ref{eg:chain-rule-without-leibniz} on page \pageref{eg:chain-rule-without-leibniz}
expressed the chain rule without the Leibniz notation, writing a function $f$ defined by
$f(x)=g(h(x))$. Suppose that you're trying to remember the rule, and two of the possibilities
that come to mind are $f'(x)=g'(h(x))$ and $f'(x)=g'(h(x))h(x)$. Show that neither of these
can possibly be right, by considering the case where $x$ has units. You may find it helpful
to convert both expressions back into the Leibniz notation.
\end{hwwithsoln}

\begin{hwwithsoln}{resonance}
When you tune in a radio station using an old-fashioned rotating dial you don't have to be
exactly tuned in to the right frequency in order to get the station. If you did, the
tuning would be infinitely sensitive, and you'd never be able to receive any signal at
all! Instead, the tuning has a certain amount of ``slop'' intentionally designed into it.
The strength of the received signal $s$ can be expressed in terms of the dial's setting
$f$ by a function of the form
\begin{equation*}
  s = \frac{1}{\sqrt{a(f^2-f_\zu{o}^2)^2+bf^2}} \qquad ,
\end{equation*}
where $a$, $b$, and $f_\zu{o}$ are constants. This functional form is in fact very general, and
is encountered in many other physical contexts. The graph below shows the resulting bell-shaped
curve. Find the frequency $f$ at which the maximum response occurs, and show that if $b$ is small,
the maximum occurs close to, but not exactly at, $f_\zu{o}$.
\end{hwwithsoln}

%%graph%% resonance func=1/sqrt(3*(x**2-1)**2+x**2) xlo=0 xhi=3 ylo=0 yhi=1.1 with=lines xtic_spacing=1 ytic_spacing=.5 format=eps
\smallfig{resonance}{The function of problem \ref{hw:resonance}, with $a=3$, $b=1$, and $f_\zu{o}=1$.}%

\widefig[t]{hw-near-focal-point}{Problem \ref{hw:near-focal-point}. A set of light rays is emitted from
the tip of the glamorous movie star's nose on the film, and reunited to form a spot on the screen which is the image of the
same point on his nose. The distances have been distorted for
clarity. The distance $y$ represents the entire length of the theater from front to back.}
%
\begin{hwwithsoln}{near-focal-point}
In a movie theater, the image on the screen is formed by a lens in the projector, and originates from one of the frames on the strip of celluloid film (or,
in the newer digital projection systems, from a liquid crystal chip).
Let the distance from the film to the lens be $x$, and let the distance from the lens to the screen be $y$. The projectionist
needs to adjust $x$ so that it is properly matched with $y$, or else the image will be out of focus. There is therefore
a fixed relationship between $x$ and $y$, and this relationship is of the form
\begin{equation*}
  \frac{1}{x}+\frac{1}{y} = \frac{1}{f} \qquad ,
\end{equation*}
where $f$ is a property of the lens, called its focal length. A stronger lens has a shorter focal length.
Since the theater is large, and the projector is relatively small, $x$ is much less than $y$.
We can see from the equation that if $y$ is sufficiently large, the left-hand side of the equation
is dominated by the $1/x$ term, and we have $x \approx f$. Since the $1/y$ term doesn't completely
vanish, we must have $x$ slightly greater than $f$, so that the $1/x$ term is slightly less than
$1/f$. Let $x=f+\der x$, and approximate $\der x$ as being infinitesimally small.
Find a simple expression for $y$ in terms of $f$ and $\der x$.
\end{hwwithsoln}

\begin{hwwithsoln}{one-to-power-infinity}
Why might the expression $1^\infty$ be considered an indeterminate form?
\end{hwwithsoln}

\end{hwsection}
