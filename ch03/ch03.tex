%%chapter%% 03
\chapter{Integration}

\section{Definite and indefinite integrals}

Because any formula can be differentiated symbolically to find another formula,
the main motivation for doing derivatives numerically would be if the function to be
differentiated wasn't known in symbolic form.
A typical example might be a two-person network computer game, in which player A's
computer needs to figure out player B's velocity based on knowledge of how her
position changes over time. But in most cases, it's numerical integration that's
interesting, not numerical differentiation.

As a warm-up, let's see how to do a running sum of a discrete function using Yacas.
The following program computers the sum $1+2+\ldots+100$ discussed to on page
\pageref{gauss-story}. Now that we're writing real computer programs with Yacas, it would be a good
idea to enter each program into a file before trying to run it. In fact, some
of these examples won't run properly if you just start up Yacas and type them in
one line at a time. If you're using Adobe Reader to read this book, you can
do \verb@Tools>Basic>Select@, select the program, copy it into a file,
and then edit out the line numbers.

\restartLineNumbers
\begin{eg}
\startcodeeg
\begin{Code}
  \nn n := 1;
  \nn sum := 0;
  \nn While (n<=100) [
  \nn   sum := sum+n;
  \nn   n := n+1;
  \nn ];
  \nn Echo(sum);
\end{Code}
\end{eg}

The semicolons are to separate one instruction from the next, and they become necessary
now that we're doing real programming.
Line 1 of this program defines the variable \verb@n@, which will take on all the values
from 1 to 100. Line 2 says that we haven't added anything up yet, so our running
sum is zero do far. Line 3 says to keep on repeating the instructions inside the
square brackets until \verb@n@ goes past 100. Line 4 updates the running sum, and
line 5 updates the value of \verb@n@. If you've never done any programming before,
a statement like \verb@n:=n+1@ might seem like nonsense --- how can a number equal
itself plus one? But that's why we use the \verb@:=@ symbol; it says that we're redefining
\verb@n@, not stating an equation. If \verb@n@ was previously 37, then after this statement is
executed, \verb@n@ will be redefined as 38.
To run the program on a Linux computer,
do this (assuming you saved the program in a file named \verb@sum.yacas@):

\begin{Code}
  \ii % yacas -pc sum.yacas 
  \oo{5050}
\end{Code}

Here the \verb@%@ symbol is the computer's prompt. The result is 5,050, as expected.
One way of stating this result is
\begin{equation*}
  \sum_{n=1}^{100} n = 5050 \qquad .
\end{equation*}
The capital Greek letter $\Sigma$, sigma, is used because it makes the ``s'' sound, and
that's the first sound in the word ``sum.'' The $n=1$ below the sigma says the sum starts
at 1, and the 100 on top says it ends at 100. The $n$ is what's known as a dummy variable:
it has no meaning outside the context of the sum. Figure \figref{discrete-sum} shows the
graphical interpretation of the sum: we're adding up the areas of a series of rectangular
strips. (For clarity, the figure only shows the sum going up to 7, rather than 100.)

\smallfig{discrete-sum}{Graphical interpretation of the sum $1+2+\ldots+7$.}

Now how about an integral? Figure \figref{integral-of-t} shows the graphical interpretation
of what we're trying to do: find the area of the shaded triangle.
This is an example we know how to do symbolically, so we can do it numerically as well,
and check the answers against each other. Symbolically, the area is given by the integral.
To integrate the function $\xdot(t)=t$, we know we need some function with a $t^2$ in it,
since we want something whose derivative is $t$, and differentiation reduces the power
by one. The derivative of $t^2$ would be $2t$ rather than $t$, so what we want is
$x(t)=t^2/2$. Let's compute the area of the triangle that stretches along the $t$ axis
from 0 to 100: $x(100)=100^/2=5000$.

\smallfig{integral-of-t}{Graphical interpretation of the integral of the function $\xdot(t)=t$.}

Figure \figref{riemann} shows how to accomplish the same thing numerically. We break up the area
into a whole bunch of very skinny rectangles. Ideally, we'd like to make the width of each
rectangle be an infinitesimal number $\der x$, so that we'd be adding up an infinite number of
infinitesimal areas. In reality, a computer can't do that, so we divide up the interval from
$t=0$ to $t=100$ into $H$ rectangles, each with finite width $\der t=100/H$. Instead of making
$H$ infinite, we make it the largest number we can without making the computer take too
long to add up the areas of the rectangles.\index{integration!computer-aided!numerical}

\smallfig{riemann}{Approximating the integral numerically.}

\restartLineNumbers
\begin{eg}\label{eg:riemann}
\startcodeeg
\begin{Code}
  \nn tmax := 100;
  \nn H := 1000;
  \nn dt := tmax/H;
  \nn sum := 0;
  \nn t := 0;
  \nn While (t<=tmax) [
  \nn   sum := N(sum+t*dt);
  \nn   t := N(t+dt);
  \nn ];
  \nn Echo(sum);
\end{Code}
\end{eg}

In example \ref{eg:riemann}, we split the interval from $t=0$ to 100 into $H=1000$ small
intervals, each with width $\der t=0.1$. The result is 5,005, which agrees with the symbolic
result to three digits of precision. Changing $H$ to 10,000 gives $5,000.5$, which is one
more digit. Clearly as we make the number of rectangles greater and greater, we're converging
to the correct result of 5,000.

In the Leibniz notation,\index{Leibniz notation!integral}
the thing we've just calculated, by two different techniques, is written like
this:
\begin{equation*}
  \int_{0}^{100} t \:\der t = 5,000
\end{equation*}
It looks a lot like the $\Sigma$ notation, with the $\Sigma$ replaces by a flattened-out letter ``S.''
The $t$ is a dummy variable. What I've been casually referring to as an integral is really two different
but closely related things, known as the definite integral and the indefinite integral.

\begin{important}[Definition of the indefinite integral]\index{integral!indefinite!definition}
If $\xdot$ is a function, then a function $x$ is an indefinite integral of $\xdot$ if,
as implied by the notation, $\der x/\der t=\xdot$.

Interpretation: Doing an indefinite integral means doing the opposite of differentiation.
All the possible indefinite integrals are the same function except for an additive constant.
\end{important}

\begin{eg}
\egquestion Find the indefinite integral of the function $\xdot(t)=t$.

\eganswer Any function of the form
\begin{equation*}
  x(t)=t^2/2+c \qquad ,
\end{equation*}
where $c$ is a constant, is an
indefinite integral of this function, since its derivative is $t$.
\end{eg}

\begin{important}[Definition of the definite integral]\index{integral!definite!definition}
If $\xdot$ is a function, then the definite integral of $\xdot$ from $a$ to $b$ is
defined as
\begin{multline*}
  \int_{a}^{b} \xdot(t) \der t \\
= \lim_{H\rightarrow\infty} \sum_{i=0}^{H} \xdot\left(a+i\Delta t\right)\Delta t \qquad ,
\end{multline*}
where $\Delta t=(b-a)/H$.

Interpretation: What we're calculating is
the area under the graph of $\xdot$, from $a$ to $b$. (If the graph dips below the $t$ axis, we interpret the
area between it and the axis as a negative area.)
The thing inside the limit is a calculation like the one done in example \ref{eg:riemann},
but generalized to $a\ne 0$.
If $H$ was infinite, then $\Delta t$ would be an infinitesimal number $\der t$. 
\end{important}

\section{The fundamental theorem of calculus}\index{calculus!fundamental theorem of!statement}\index{fundamental theorem of calculus!statement}

\begin{important}[The fundamental theorem of calculus]
Let $x$ be an indefinite integral of $\xdot$, and let $\xdot$ be a continuous
function (one whose graph is a single connected curve). Then

\begin{equation*}
  \int_{a}^{b} \xdot(t) \der t = x(b)-x(a) \qquad .
\end{equation*}

Interpretation: In the simple examples we've been doing so far,
we were able to choose an indefinite integral such that $x(0)=0$.
In that case, $x(t)$ is interpreted as the area from 0 to $t$,
so in the expression $x(b)-x(a)$, we're taking the area from
0 to $a$, but subtracting out the area from 0 to $b$, which gives
the area from $a$ to $b$. If we choose an indefinite integral with
a different $c$, the $c$'s will just cancel out anyway in the
difference  $x(b)-x(a)$.
\end{important}

The fundamental theorem is proved on page \pageref{detour:fundamental-thm-proof}.

\begin{eg}\label{eg:hyperbola}
\egquestion Interpret the indefinite integral
\begin{equation*}
  \int_{1}^{2} \frac{1}{t}\:\der t \qquad .
\end{equation*}
graphically; then evaluate it it both symbolically and numerically, and check that the two results
are consistent.

\smallfig{hyperbola-area}{The indefinite integral $\int_1^2 (1/t)\der t$.}

\eganswer Figure \figref{hyperbola-area} shows the graphical interpretation. The numerical calculation
requires a trivial variation on the program from example \ref{eg:riemann}:

\restartLineNumbers
\begin{Code}
  a := 1;
  b := 2;
  H := 1000;
  dt := (b-a)/H;
  sum := 0;
  t := a;
  While (t<=b) [
    sum := N(sum+(1/t)*dt);
    t := N(t+dt);
  ];
  Echo(sum);
\end{Code}
The result is $0.693897243$, and increasing $H$ to 10,000 gives $0.6932221811$, so we can be fairly confident that
the result equals $0.693$, to 3 decimal places.

Symbolically, the indefinite integral is $x=\ln t$. Using the fundamental theorem of calculus, the area
is $\ln 2-\ln 1\approx 0.693147180559945$.

Judging from the graph, it looks plausible that the shaded area is about 0.7.

This is an interesting example, because the natural log blows up to negative infinity as $t$ approaches 0,
so it's not possible to add a constant onto the indefinite integral and force it to be equal to
0 at $t=0$. Nevertheless, the fundamental theorem of calculus still works.

\end{eg}

%%graph%% hyperbola-area-raw func=1/x format=svg xlo=0 xhi=3 ylo=0 yhi=1.5 with=lines x=t y=x xtic_spacing=1 ytic_spacing=1

\section{Properties of the integral}\index{integral!properties of}

Let $f$ and $g$ be two functions of $x$, and let $c$ be a constant. We already know that for derivatives,
\begin{gather*}
  \frac{\der}{\der x}(f+g) = \frac{\der f}{\der x} + \frac{\der g}{\der x} \\
\intertext{and}
  \frac{\der}{\der x}(cf) = c\frac{\der f}{\der x} \qquad .
\end{gather*}
But since the indefinite integral is just the operation of undoing a derivative, the
same kind of rules must hold true for indefinite integrals as well:
\begin{gather*}
  \int(f+g)\der x =   \int f \der x +   \int g \der x \\
\intertext{and}
  \int(cf)\der x =   c \int f \der x \qquad .
\end{gather*}
And since a definite integral can be found by plugging in the upper and lower limits of integration
into the indefinite integral, the same properties must be true of definite integrals as well.

\begin{eg}
\egquestion Evaluate the indefinite integral
\begin{equation*}
  \int (x+2\sin x) \der x \qquad .
\end{equation*}

\eganswer
Using the additive property, the integral becomes
\begin{equation*}
 \int x \der x +   \int 2\sin x \der x \qquad .
\end{equation*}
Then the property of scaling by a constant lets us change this to
\begin{equation*}
                             \int x \der x +   2 \int \sin x \der x \qquad .\\
\end{equation*}
We need a function whose derivative is $x$, which would be $x^2/2$, and
one whose derivative is $\sin x$, which must be $-\cos x$, so the result is
\begin{equation*}
                             \frac{1}{2}x^2 -   2 \cos x + c \qquad .
\end{equation*}
\end{eg}

\section{Applications}

\subsection{Averages}\index{average}

In the story of Gauss's problem of adding up the numbers from 1 to 100, one interpretation of
the result, 5,050, is that the average of all the numbers from 1 to 100 is 50.5. This is the
ordinary definition of an average: add up all the things you have, and divide by the number
of things. (The result in this example makes sense, because half the numbers
are from 1 to 50, and half are from 51 to 100, so the average is half-way between 50 and 51.)

Similarly, a definite integral can also be thought of as a kind of average. In general, if $y$ is
a function of $x$, then the average, or mean, value of $y$ on the interval from $x=a$ to $b$ can be
defined as
\begin{equation*}
  \bar{y} = \frac{1}{b-a} \int_a^b y\:\der x \qquad .
\end{equation*}
In the continuous case, dividing by $b-a$ accomplishes the same thing as dividing by the
number of things in the discrete case.

\begin{eg}
\egquestion Show that the definition of the average makes sense in the case where the function
is a constant.

\eganswer If $y$ is a constant, then we can take it outside of the integral, so
\begin{align*}
  \bar{y} &= \frac{1}{b-a} y \int_a^b 1\:\der x \\
                     &= \frac{1}{b-a} y \left.x\right|_a^b \\
                     &= \frac{1}{b-a} y (b-a) \\
                     &= y
\end{align*}
\end{eg}

\begin{eg}\label{eg:mean-of-square}
\egquestion Find the average value of the function $y=x^2$ for values of $x$ ranging from
0 to 1.

\begin{align*}
  \bar{y} &= \frac{1}{1-0} \int_0^1 x^2\:\der x \\
                     &= \left.\frac{1}{3}x^3\right|_0^1 \\
                     &= \frac{1}{3}
\end{align*}
\end{eg}

\begin{important}[The mean value theorem]\index{mean value theorem!statement}
If the continuous function $y(x)$ has the average value $\bar{y}$ on the interval
from $x=a$ to $b$, then $y$ attains its average value at least once in that interval,
i.e., there exists $\xi$ with $a<\xi<b$ such that $y(\xi)=\bar{y}$.
\end{important}

The mean value theorem is proved on page \pageref{detour:mean-value-proof}.
The special case in which $\bar{y}=0$ is known as Rolle's theorem.\index{Rolle's theorem}

\begin{eg}
\egquestion Verify the mean value theorem for $y=x^2$ on the interval from 0 to 1.

\eganswer The mean value is 1/3, as shown in example \ref{eg:mean-of-square}. This value
is achieved at $x=\sqrt{1/3}=1/\sqrt{3}$, which lies between 0 and 1.
\end{eg}

\subsection{Work}\index{work}
In physics, work is a measure of the amount of energy transferred by a force; for example, if a horse
sets a wagon in motion, the horse's force on the wagon is putting some energy of motion into the wagon.
When a force $F$ acts on an object that moves in the direction of the force by an infinitesimal
distance $\der x$, the infinitesimal work done is $\der W=F\der x$. Integrating both sides,
we have $W=\int_a^b F\der x$, where the force may depend on $x$, and $a$ and $b$ represent the
initial and final positions of the object. 

\begin{eg}
\egquestion A spring compressed by an amount $x$ relative to its relaxed length provides
a force $F=kx$. Find the amount of work that must be done in order to compress the spring
from $x=0$ to $x=a$. (This is the amount of energy stored in the spring, and that energy
will later be released into the toy bullet.)

\eganswer
\begin{align*}
  W &= \int_0^a F \der x \\
    &= \int_0^a kx \der x
\end{align*}
\begin{align*}
    &= \left.\frac{1}{2}kx^2\right|_0^a \\
    &= \frac{1}{2}ka^2
\end{align*}
The reason $W$ grows like $a^2$, not just like $a$, is that as the spring is compressed
more, more and more effort is required in order to compress it.
\end{eg}

\subsection{Probability}

Mathematically, the probability that something will happen can be specified with a number
ranging from 0 to 1, with 0 representing impossibility and 1 representing certainty.
If you flip a coin, heads and tails both have probabilities of 1/2.
The sum of the probabilities of all the possible outcomes has to have probability 1.
This is called \emph{normalization}.\index{probability}\index{normalization}

\smallfig{globe}{Normalization: the probability of picking land plus
the probability of picking water adds up to 1.}

So far we've discussed random processes having only two
possible outcomes: yes or no, win or lose, on or off. More
generally, a random process could have a result that is a
number. Some processes yield integers, as when you roll a
die and get a result from one to six, but some are not
restricted to whole numbers, e.g., the height of a human being, or
the amount of time that a uranium-238 atom will exist before undergoing radioactive decay.
The key to handling these continuous random variables is the concept of the area
under a curve, i.e., an integral.
%
\fig{single-die}{Probability distribution for the result of rolling a single die.}

Consider a throw of a die. If the die is ``honest,'' then we
expect all six values to be equally likely. Since all six
probabilities must add up to 1, then probability of any
particular value coming up must be 1/6. We can summarize
this in a graph, \figref{single-die}. Areas under the curve can be
interpreted as total probabilities. For instance, the area
under the curve from 1 to 3 is $1/6+1/6+1/6=1/2$, so the
probability of getting a result from 1 to 3 is 1/2. The
function shown on the graph is called the probability distribution.
%
\fig{two-dice}{Rolling two dice and adding them up.}

Figure \figref{two-dice} shows the probabilities of various results
obtained by rolling two dice and adding them together, as in
the game of craps. The probabilities are not all the same.
There is a small probability of getting a two, for example,
because there is only one way to do it, by rolling a one and
then another one. The probability of rolling a seven is high
because there are six different ways to do it: 1+6, 2+5, etc.

If the number of possible outcomes is large but finite, for
example the number of hairs on a dog, the graph would start
to look like a smooth curve rather than a ziggurat.

What about probability distributions for random numbers that
are not integers? We can no longer make a graph with
probability on the $y$ axis, because the probability of
getting a given exact number is typically zero. For
instance, there is zero probability that a person will be
\emph{exactly} 200 cm tall, since there are
infinitely many possible results that are close to 200 but not
exactly two, for example 199.99999999687687658766.
It doesn't usually make sense, therefore, to talk
about the probability of a single numerical result, but it
does make sense to talk about the probability of a certain
range of results. For instance, the probability that a randomly
chosen person will be more than 170 cm and less than 200 cm
in height is a perfectly
reasonable thing to discuss. We can still summarize the
probability information on a graph, and we can still
interpret areas under the curve as probabilities.
%
\fig{human-height}{A probability distribution for human height.}

But the $y$ axis can no longer be a unitless probability
scale. In the example of human height, we want the $x$
axis to have units of meters, and we want areas under the
curve to be unitless probabilities. The area of a single
square on the graph paper is then
\begin{align*}
 (\text{unitless area of a square})\\
         =  (\text{width of square}\\
             \text{with distance units}) \\
             \times (\text{height of square}) \qquad .
\end{align*}
If the units are to cancel out, then the height of the
square must evidently be a quantity with units of inverse
centimeters. In other words, the $y$ axis of the graph is to be
interpreted as probability per unit height, not probability.

Another way of looking at it is that the $y$ axis on the graph
gives a derivative, $\der P/\der x$: the infinitesimally small
probability that $x$ will lie in the infinitesimally small
range covered by $\der x$.

\begin{eg}\label{eg:random-diff}
\egquestion A computer language will typically have a built-in
subroutine that produces a fairly random number that is equally
likely to take on any value in the range from 0 to 1. If you
take the absolute value of the difference between two such
numbers, the probability distribution is of the form
$\der P/\der x=k(1-x)$. Find the value of the constant $k$
that is required by normalization.

\eganswer
\begin{align*}
  1 &= \int_0^1 k(1-x)\:\der x \\
    &= \left.kx-\frac{1}{2}kx^2\right|_0^1 \\
    &= k-k/2 \\
  k &= 2
\end{align*}
\end{eg}

\begin{selfcheck}{interpret-human-height}
Compare the number of people with heights in the range of
130-135 cm to the number in the range 135-140.
\end{selfcheck}

\fig{average}{The average can be interpreted as the balance point
of the probability distribution.}

When one random variable is related to another in some mathematical
way, the chain rule can be used to relate their probability distributions.

\fig{laser}{Example \ref{eg:laser}.}

\begin{eg}\label{eg:laser}
\egquestion A laser is placed one meter away from a wall, and spun
on the ground to give it a random direction, but if the angle $u$ shown in figure
\figref{laser} doesn't
come out in the range from 0 to $\pi/2$, the laser is spun again until
an angle in the desired range is obtained. Find the probability distribution
of the distance $x$ shown in the figure. The derivative
$\der\tan^{-1}z/\der z=1/(1+z^2)$ will be required (see example \ref{eg:arctan},
page \pageref{eg:arctan}).

\eganswer Since any angle between 0 and $\pi/2$ is equally likely, the
probability distribution $\der P/\der u$ must be a constant, and normalization
tells us that the constant must be $\der P/\der u=2/\pi$. 

The laser is one meter from the wall, so the distance $x$, measured in
meters, is given by $x=\tan u$. For the probability distribution of $x$, we have
\begin{align*}
  \frac{\der P}{\der x} &= \frac{\der P}{\der u}\cdot\frac{\der u}{\der x} \\
        &= \frac{2}{\pi}\cdot\frac{\der \tan^{-1}x}{\der x} \\
        &= \frac{2}{\pi(1+x^2)}
\end{align*}

Note that the range of possible values of $x$ theoretically extends from 0 to
infinity. Problem \ref{hw:laser} on page \pageref{hw:laser} deals with this.
\end{eg}

If the next Martian you meet asks you, ``How tall is an
adult human?,'' you will probably reply with a statement
about the average human height, such as ``Oh, about 5 feet 6
inches.'' If you wanted to explain a little more, you could
say, ``But that's only an average. Most people are somewhere
between 5 feet and 6 feet tall.'' Without bothering to draw
the relevant bell curve for your new extraterrestrial
acquaintance, you've summarized the relevant information by
giving an average and a typical range of variation.
The average of a probability distribution can be defined
geometrically as the horizontal position at which it could
be balanced if it was constructed out of cardboard, \figref{average}.
This is a different way of working with averages than the one
we did earlier. Before, had a graph of $y$ versus $x$, we implicitly
assumed that all values of $x$ were equally likely, and we found an
average value of $y$. In this new method using probability distributions,
the variable we're averaging is on the $x$ axis, and the $y$ axis
tells us the relative probabilities of the various $x$ values.

For a discrete-valued variable with $n$ possible values, the average would be
\begin{equation*}
  \bar{x} = \sum_{i=0}^n x\:P(x) \qquad ,
\end{equation*}
and in the case of a continuous variable, this becomes an integral,
\begin{equation*}
  \bar{x} = \int_a^b x\:\frac{\der P}{\der x} \: \der x \qquad .
\end{equation*}

\begin{eg}
\egquestion For the situation described in example \ref{eg:random-diff},
find the average value of $x$.

\eganswer
\begin{align*}
  \bar{x} &= \int_0^1 x\:\frac{\der P}{\der x} \: \der x \\
          &= \int_0^1 x \cdot 2(1-x)\:\der x \\
          &= 2\int_0^1 (x-x^2)\:\der x \\
          &= 2\left. \left(\frac{1}{2}x^2-\frac{1}{3}x^3\right) \right|_0^1 \\
          &= \frac{1}{3}
\end{align*}
\end{eg}

Sometimes we don't just want to know the average value of a certain variable, we
also want to have some idea of the amount of variation above and below the average.
The most common way of measuring this is the \emph{standard deviation},\index{standard deviation}
defined by
\begin{equation*}
  \sigma = \sqrt{\int_a^b (x-\bar{x})^2\:\frac{\der P}{\der x} \: \der x} \qquad .
\end{equation*}
The idea here is that if there was no variation at all above or below the average,
then the quantity $(x-\bar{x})$ would be zero whenever $\der P/\der x$ was nonzero, and
the standard deviation would be zero. The reason for taking the square root of the whole
thing is so that the result will have the same units as $x$.

\begin{eg}
\egquestion For the situation described in example \ref{eg:random-diff},
find the standard deviation of $x$.

\eganswer The square of the standard deviation is
\begin{align*}
  \sigma^2 &= \int_0^1 (x-\bar{x})^2\:\frac{\der P}{\der x} \: \der x \\
          &= \int_0^1 (x-1/3)^2\cdot2(1-x) \: \der x \\
          &= 2 \int_0^1 \left(-x^3+\frac{5}{3}x^2-\frac{7}{9}x+\frac{1}{9}\right) \: \der x \\
          &= \frac{1}{18} \qquad ,
\intertext{so the standard deviation is}
  \sigma &= \frac{1}{\sqrt{18}} \\
         &\approx 0.236
\end{align*}
\end{eg}

%---------------------------------------------------------------------------------

\begin{hwsection}
\begin{hwwithsoln}{integrate-num}
Write a computer program similar to the one in example \ref{eg:hyperbola} on page \pageref{eg:hyperbola}
to evaluate the definite integral
\begin{equation*}
  \int_0^1 e^{x^2} \qquad .
\end{equation*}
\end{hwwithsoln}

\begin{hwwithsoln}{integrate-sin-cancel}
Evaluate the integral
\begin{equation*}
  \int_0^{2\pi} \sin x \:\der x \qquad ,
\end{equation*}
and draw a sketch to explain why your result comes out the way it does.
\end{hwwithsoln}

\begin{hwwithsoln}{estimate-then-integrate}
Sketch the graph that represents the definite integral
\begin{equation*}
  \int_0^2 -x^2+2x \qquad ,
\end{equation*}
and estimate the result roughly from the graph. Then evaluate the
integral exactly, and check against your estimate.
\end{hwwithsoln}

\begin{hwwithsoln}{average-sine}
Make a rough guess as to the average value of $\sin x$ for $0<x<\pi$, and then
find the exact result and check it against your guess.
\end{hwwithsoln}

\begin{hwwithsoln}{continuity-mean-value}
Show that the mean value theorem's assumption of continuity is necessary, by exhibiting
a discontinuous function for which the theorem fails.
\end{hwwithsoln}

\begin{hwwithsoln}{fund-thm-assumption}
Show that the fundamental theorem of calculus's assumption of continuity for $\xdot$ is necessary, by exhibiting
a discontinuous function for which the theorem fails.
\end{hwwithsoln}

\begin{hw}
Sketch the graphs of $y=x^2$ and $y=\sqrt{x}$ for $0\le x\le 1$. Graphically, what relationship
should exist between the integrals $\int_0^1 x^2\:\der x$ and $\int_0^1 \sqrt{x}\:\der x$? Compute both
integrals, and verify that the results are related in the expected way.
\end{hw}

\begin{hw}\label{hw:piston}
In a gasoline-burning car engine, the exploding air-gas
mixture makes a force on the piston, and the force tapers off as the piston expands, allowing the
gas to expand. (a) In the approximation $F=k/x$, where $x$ is the position of the piston, find the
work done on the piston as it travels from $x=a$ to $x=b$, and show that the result only depends
on the ratio $b/a$. This ratio is known as the compression ratio of the engine. (b) A better
approximation, which takes into account the cooling of the air-gas mixture as it expands, is
$F=kx^{-1.4}$. Compute the work done in this case.
\end{hw}

\smallfig{hw-piston}{Problem \ref{hw:piston}.}

\begin{hw}
A certain variable $x$ varies randomly from -1 to 1, with probability distribution
$\der P/\der x=k\left(1-x^2\right)$.\\
(a) Determine $k$ from the requirement of normalization.\\
(b) Find the average value of $x$.\\
(c) Find its standard deviation.
\end{hw}

\begin{hw}[2]
A perfectly elastic ball bounces up and down forever, always coming back up to the same height $h$. Find
its average height.
\end{hw}

\fig{hw-holditch}{Problem \ref{hw:holditch}.}

\begin{hw}[2]\label{hw:holditch}
The figure shows a curve with a tangent line segment of length 1 that sweeps around it, forming a new curve
that is usually outside the old one.
Prove Holditch's theorem, which states that the  new curve's area differs from the old one's
by $\pi$.\index{Holditch's theorem}
(This is an example of a result that is much more
difficult to prove without making use of infinitesimals.)
\end{hw}

\end{hwsection}
