%%chapter%% 05
\chapter{Sequences and Series}

\section{Infinite sequences}\index{sequence}

Consider an infinite sequence of numbers like $1/2$, $2/3$, $3/4$, $4/5$, \ldots We want to define
this as approaching 1, or ``converging to 1.'' The way to do this is to make a function $f(n)$, which is only
well defined for integer values of $n$. Then $f(1)=1/2$, $f(2)=2/3$, and in general
$f(n)=n/(n+1)$. With just a little tinkering, our definitions of limits can be
applied to this type of function  (see problem \ref{hw:sequence-weierstrass} on page \pageref{hw:sequence-weierstrass}).

\section{Infinite series}
A related question is how to rigorously define the sum of infinitely many numbers, which is referred to
as an infinite \emph{series}.\index{series!infinite} An example is the geometric series
$1+x+x^2+x^3+\ldots=1/(1-x)$,\index{series!geometric}\index{geometric series}\index{Euclid}
which we used casually on page \pageref{geometric-series}.
The general concept of an infinite series goes back
to ancient Greek mathematics. Various supposed paradoxes about infinite series, such as Zeno's paradox,\index{Zeno's paradox}
were exhibited, influencing Euclid to sidestep the issue in his \emph{Elements}, where in Book IX, Proposition 35
he provides only an expression $(1-x^n)/(1-x)$ for the $n$th partial sum of the geometric series.
The case where $n$ gets so big that $x^n$ becomes negligible is left to the reader's imagination,
as in one of those scenes in a romance novel that ends with something like ``...and she surrendered...''
For those with modern training, the idea is that an infinite sum like $1+1+1+\ldots$ would
clearly give an infinite result, but this is only because the terms are all staying the same size. If the terms get
smaller and smaller, and get smaller fast enough, then the result can be finite. For example, consider the geometric
series in the case where $x=1/2$, for which we expect the result $1/(1-1/2)=2$. We have
\begin{equation*}
  1 + \frac{1}{2} + \frac{1}{4} + \frac{1}{8} + \frac{1}{16} + \ldots \qquad ,
\end{equation*}
which at the successive steps of addition equals 1, $1\frac{1}{2}$, $1\frac{3}{4}$, $1\frac{7}{8}$,  $1\frac{15}{16}$, \ldots.
We're getting closer and closer to 2, cutting the distance in half at each step. Clearly we can get as close as we like to 2,
if we're willing to add enough terms.

Note that we ended up wanting to talk about the partial sums of the series. This is the right way to get a rigorous definition of
the convergence of series in general.
In the case of the geometric series, for example,
we can define a sequence of the
partial sums 1, $1+x$, $1+x+x^2$, \ldots We can then define convergence and limits of series in terms of convergence and limits
of the partial sums.

It's instructive to see what happens to the geometric series with $x=0.1$. The geometric series becomes\label{geometric-tenths}
\begin{equation*}
  1 + 0.1 + 0.01 + 0.001 + \ldots \qquad .
\end{equation*}
The partial sums are 1, 1.1, 1.11, 1.111, \ldots We can see vividly here that adding another term will only affect
the result in a certain decimal place, without affecting any of the earlier ones. For instance, if we needed a
result that was valid to three digits past the decimal place, we could stop at 1.111, being assured that we had
attained a good enough approximation. If we wanted an exact result, we could also observe that multiplying the
result by 9 would give $9.999\ldots$, which is the same as 10, so the result must be 10/9, which is in agreement
with $1/(1-1/10)=10/9$.

One thing to watch out for with infinite series is that
the axioms of the real number system only talk about finite sums, so it's easy to get wrong results
by attempting to apply them to infinite ones (see problem
\ref{hw:bogus-series} on page \pageref{hw:bogus-series}).\label{infinite-sum-warning}

\section{Tests for convergence}

There are many different tests that can be used to determine whether a sequence or series converges.
I'll briefly state three of the most useful, with sketches of their proofs.

\emph{Bounded and increasing sequences\/:} A sequence that always increases, but never
surpasses a certain value, converges.

This amounts to a restatement of the compactness axiom for the real numbers stated on page \pageref{compactness},
and is therefore to be interpreted not so much as a statement about sequences but as one about the
real number system. In particular, it fails if interpreted as a statement about sequences confined entirely to the rational number system, as we can see from the
sequence 1, 1.4, 1.41, 1.414, \ldots consisting of the successive decimal approximations to $\sqrt{2}$,
which does not converge to any rational-number value.

\begin{eg}
\egquestion Prove that the geometric series
$1+1/2+1/4+\ldots$ converges.

\eganswer The sequence of partial sums is increasing, since each term is positive.
Each term closes half of the remaining gap separating the previous partial sum from 2,
so the sum never surpasses 2. Since the partial sums are increasing and bounded,
they converge to a limit.
\end{eg}

Once we know that a particular series converges, we can also easily infer the convergence
of other series whose terms get smaller faster. For example, we can be certain
that if the geometric series converges, so does the series
\begin{equation*}
  \frac{1}{1} + \frac{1}{1\times 2} + \frac{1}{1\times 2 \times 3} +\ldots \qquad ,
\end{equation*}
whose terms get smaller faster than any base raised to the power $n$.

\emph{Alternating series with decreasing terms\/:} If the terms of a series alternate
in sign and approach zero, then the series converges.

Sketch of a proof: The even partial sums form an increasing sequence, the odd sums a decreasing one.
Neither of these sequences of partial sums can be unbounded, since the difference between partial
sums $n$ and $n+1$ would then have to be unbounded, but this difference is simply the $n$th term,
and the terms approach zero. Since the even
partial sums are increasing and bounded, they converge to a limit, and similarly for the odd ones.
The two limits must be equal, since the terms approach zero.

\begin{eg}
\egquestion Prove that the series $1-1/2+1/3-1/4+\ldots$ converges.

\eganswer Its convergence follows because it is an alternating series with decreasing terms.
The sum turns out to be $\ln 2$, although the convergence of the series is so slow that
an extremely large number of terms is required in order to obtain a decent approximation,
\end{eg}

\emph{The integral test\/:} If the terms of a series $a_n$ are positive and decreasing, and $f(x)$ is a positive and decreasing function
on the real number line such that $f(n)=a_n$, then the sum of $a_n$ from $n=1$ to $\infty$ converges if and only if
$\int_1^\infty f(x)\der x$ does.

Sketch of proof: Since the theorem is supposed to hold for both convergence and divergence, and is also an ``if and only if,''
there are actually four cases to prove, of which we pick the representative one where the integral is known to converge and we
want to prove convergence of the corresponding sum.
The sum and the integral can be interpreted as the areas under two graphs: one like a smooth ramp and one like a staircase.
Sliding the staircase half a unit to the left, it lies entirely underneath the ramp, and therefore the area under it is
also finite.

\begin{eg}
\egquestion Prove that the series $1+1/2+1/3+\ldots$ diverges.

\eganswer The integral of $1/x$ is $\ln x$, which diverges as $x$ approaches infinity, so the series diverges as well.
\end{eg}


\section{Taylor series}\label{sec:taylor}\index{Taylor series}\index{series!Taylor}

If you calculate $e^{0.1}$ on your calculator, you'll find that
it's very close to 1.1. This is because the tangent line at $x=0$
on the graph of $e^x$ has a slope of 1 ($\der e^x/\der x=e^x=1$ at $x=0$),
and the tangent line is a good approximation to the exponential curve
as long as we don't get too far away from the point of tangency.

%%graph%% ex-tangent func=exp(x) format=eps xlo=-1 xhi=1 ylo=0 yhi=3 with=lines xtic_spacing=1 ytic_spacing=1 ; func=1+x
\smallfig{ex-tangent}{The function $e^x$, and the tangent line at $x=0$.}

How big is the error? The actual value of $e^{0.1}$ is $1.10517091807565\ldots$, which
differs from $1.1$ by about $0.005$. If we go farther from the point of tangency,
the approximation gets worse. At $x=0.2$, the error is about $0.021$, which is about
four times bigger. In other words, doubling $x$ seems to roughly quadruple the error,
so the error is proportional to $x^2$; it seems to be about $x^2/2$. Well, if we want a handy-dandy, super-accurate
estimate of $e^x$ for small values of $x$, why not just account for this
error. Our new and improved estimate is
\begin{align*}
  e^x \approx 1+x+\frac{1}{2}x^2
\end{align*}
for small values of $x$.

%%graph%% ex-quadratic func=exp(x) format=eps xlo=-1 xhi=1 ylo=0 yhi=3 with=lines xtic_spacing=1 ytic_spacing=1 ; func=1+x+x**2/2
\smallfig{ex-quadratic}{The function $e^x$, and the approximation $1+x+x^2/2$.}

Figure \figref{ex-quadratic} shows that the approximation is now extremely good for sufficiently small values of $x$.
The difference is that whereas $1+x$ matched both the y-intercept and the slope of the curve, $1+x+x^2/2$ matches
the curvature as well. Recall that the second derivative is a measure of curvature. The second derivatives of the
function and its approximation are
\begin{gather*}
  \frac{\der}{\der x} e^x = 1\\
  \frac{\der}{\der x} \left(1+x+\frac{1}{2}x^2\right) = 1
\end{gather*}

%%graph%% ex-cubic func=exp(x) format=eps xlo=-1 xhi=1 ylo=0 yhi=3 with=lines xtic_spacing=1 ytic_spacing=1 ; func=1+x+x**2/2+x**3/6
\smallfig{ex-cubic}{The function $e^x$, and the approximation $1+x+x^2/2+x^3/6$.}
%
We can do even better. Suppose we want to match the third derivatives. All the derivatives of $e^x$, evaluated
at $x=0$, are 1, so we just need to add on a term proportional to $x^3$ whose third derivative is one. Taking the
first derivative will bring down a factor of 3 in front, and taking and the second derivative will give a 2, so
to cancel these out we need the third-order term to be $(1/2)(1/3)$:
\begin{align*}
  e^x \approx 1+x+\frac{1}{2}x^2+\frac{1}{2\cdot3}x^3
\end{align*}
Figure \figref{ex-cubic} shows the result. For a significant range of $x$ values close to zero, the approximation is
now so good that we can't even see the difference between the two functions on the graph.

 On the other hand, figure \figref{ex-cubic-wide} shows that 
the cubic approximation for somewhat larger negative and positive values of $x$ is poor --- worse, in fact, than the
linear approximation, or even the constant approximation $e^x=1$. This is to be expected, because
any polynomial will blow up to either positive or negative infinity as $x$ approaches negative infinity, whereas
the function $e^x$ is supposed to get very close to zero for large negative $x$. The idea here is that derivatives
are \emph{local} things: they only measure the properties of a function very close to the point at which they're
evaluated, and they don't necessarily tell us anything about points far away.
%
%%graph%% ex-cubic-wide func=exp(x) format=eps xlo=-4 xhi=2.5 ylo=-10 yhi=12 with=lines xtic_spacing=1 ytic_spacing=5 ; func=1+x+x**2/2+x**3/6
\fig{ex-cubic-wide}{The function $e^x$, and the approximation $1+x+x^2/2+x^3/6$, on a wider scale.}

It's a remarkable fact, then, that by taking enough terms in a polynomial approximation, we can always get as good an
approximation to $e^x$ as necessary --- it's just that a large number of terms may be required for large values of $x$.
In other words, the \emph{infinite series}\index{series, infinite}
\begin{equation*}
   1+x+\frac{1}{2}x^2+\frac{1}{2\cdot3}x^3+\ldots
\end{equation*}
always gives exactly $e^x$. But what is the pattern here that would allows us to figure out, say, the fourth-order
and fifth-order terms that were swept under the rug with the symbol ``\ldots''? Let's do the fifth-order term as
an example. The point of adding in a fifth-order term is to make the fifth derivative of the approximation equal
to the fifth derivative of $e^x$, which is 1. The first, second, \ldots derivatives of $x^5$ are
\begin{align*}
  \frac{\der}{\der x}x^5 &= 5x^4 \\
  \frac{\der^2}{\der x^2}x^5 &= 5\cdot4 x^3 \\
  \frac{\der^3}{\der x^3}x^5 &= 5\cdot4\cdot3 x^2 \\
  \frac{\der^4}{\der x^4}x^5 &= 5\cdot4\cdot3\cdot2 x \\
  \frac{\der^5}{\der x^5}x^5 &= 5\cdot4\cdot3\cdot2\cdot1  
\end{align*}
The notation for a product like $1\cdot2\cdot\ldots\cdot n$ is $n!$, read\index{factorial} ``$n$ factorial.'' So to get a term
for our polynomial whose fifth derivative is 1, we need $x^5/5!$. The result for the infinite series is
\begin{equation*}
  e^x = \sum_{n=0}^\infty \frac{x^n}{n!} \qquad ,
\end{equation*}
where the special case of $0!=1$ is assumed.\footnote{This makes sense, because, for example, 4!=5!/5, 3!=4!/4, etc., so
we should have 0!=1!/1.} This infinite series is called the \emph{Taylor series} for $e^x$, evaluated around $x=0$, and it's true, although I haven't proved it,
that this particular Taylor series always converges to $e^x$, no matter how far $x$ is from zero.

In general, the Taylor series
around $x=0$ for a function $y$ is
\begin{equation*}
  T_0(x) = \sum_{n=0}^\infty a_n x^n \qquad ,
\end{equation*}
where the condition for equality of the nth order derivative is
\begin{equation*}
  \left. a_n = \frac{1}{n!}\frac{\der^n y}{\der x^n}\right|_{x=0} \qquad .
\end{equation*}
Here the notation $\left.\right|_{x=0}$ means that the derivative is to be evaluated at $x=0$.

A Taylor series can be used to approximate other functions besides $e^x$, and when you ask your calculator to evaluate a
function such as a sine or a cosine, it may actually be using a Taylor series to do it.
Taylor series are also the method Inf uses to calculate most expressions involving infinitesimals. In example
\ref{eg:geometric-series} on page \pageref{eg:geometric-series}, we saw that
when Inf was asked to calculate $1/(1-d)$, where $d$ was infinitesimal, the result was the geometric series:
\begin{Code}
  \ii : 1/(1-d)
  \oo{1+d+d^2+d^3+d^4}
\end{Code}
These are also the the first five terms of the Taylor series for the function $y=1/(1-x)$, evaluated around $x=0$.
That is, the geometric series $1+x+x^2+x^3+\ldots$ is really just one special example of a Taylor series, as
demonstrated in the following example.

\begin{eg}\label{eg:taylor-geometric}
\egquestion Find the Taylor series of $y=1/(1-x)$ around $x=0$.

\eganswer Rewriting the function as $y=(1-x)^{-1}$ and applying the chain rule, we have
\begin{align*}
                                                & \left.y\right|_{x=0} = 1\\
  \left.\frac{\der y}{\der x}\right|_{x=0}     =& \left.(1-x)^{-2}\right|_{x=0} = 1\\
  \left.\frac{\der^2 y}{\der x^2}\right|_{x=0} =& \left.2(1-x)^{-3}\right|_{x=0} = 2\\
  \left.\frac{\der^3 y}{\der x^3}\right|_{x=0} =& \left.2\cdot3(1-x)^{-4}\right|_{x=0} = 2\cdot3\\
                                                & \ldots
\end{align*}
The pattern is that the nth derivative is $n!$. The Taylor series therefore has $a_n=n!/n!=1$:
\begin{equation*}
  \frac{1}{1-x} =  1+x+x^2+x^3+\ldots
\end{equation*}
\end{eg}

If you flip back to page \pageref{geometric-tenths} and compare the rate of convergence of the geometric
series for $x=0.1$ and $0.5$, you'll see that the sum converged 
much more quickly for $x=0.1$ than for $x=0.5$. In general, we expect
that any Taylor series will converge more quickly when $x$ is smaller. Now consider what happens at $x=1$. The series
is now $1+1+1+\ldots$, which gives an infinite result, and we shouldn't have expected any better behavior, since
attempting to evaluate $1/(1-x)$ at $x=1$ gives division by zero. For $x>1$, the results become nonsense.
For example, $1/(1-2)=-1$, which is finite, but the geometric series gives $1+2+4+\ldots$, which is infinite.

In general, every function's Taylor series around $x=0$ converges for all values of $x$ in the range
defined by $|x|<r$, where $r$ is some number, known as the radius of convergence.\index{radius of convergence}
Also, if the function is defined by putting together other functions that are well behaved (in the sense of
converging to their own Taylor series in the relevant region), then the Taylor series will not only converge
but converge to the \emph{correct} value.
For the function $e^x$, the radius
happen to be infinite, whereas for $1/(1-x)$ it equals 1. The following example shows a worst-case
scenario.

%%graph%% nonanalytic func=exp(-1/x**2) format=eps xlo=-4 xhi=4 ylo=0 yhi=1.1 with=lines xtic_spacing=2 ytic_spacing=1
\fig{nonanalytic}{The function $e^{-1/x^2}$ never converges to its Taylor series.}

\begin{eg}
The function $y=e^{-1/x^2}$, shown in figure \figref{nonanalytic}, never converges to its Taylor series, except at $x=0$.
This is because the Taylor series for this function, evaluated around $x=0$ is exactly zero! At $x=0$, we have
$y=0$, $\der y/\der x=0$, $\der^2 y/\der x^2=0$, and so on for every derivative. The zero function matches the function
$y(x)$ and all its derivatives to all orders, and yet is useless as an approximation to $y(x)$. The radius of convergence
of the Taylor series is infinite, but it doesn't give correct results except at $x=0$. The reason for this is that
$y$ was built by composing two functions, $w(x)=-1/x^2$ and $y(w)=e^w$. The function $w$ is badly behaved at $x=0$
because it blows up there. In particular, it doesn't have a well-defined Taylor series at $x=0$.
\end{eg}

\begin{eg}\label{eg:taylor-sin}
\egquestion Find the Taylor series of $y=\sin x$, evaluated around $x=0$.

\eganswer  The first few derivatives are
\begin{align*}
  \frac{\der}{\der x}\sin x &= \cos x \\
  \frac{\der^2}{\der x^2}\sin x &= -\sin x\\
  \frac{\der^3}{\der x^3}\sin x &= -\cos x\\
  \frac{\der^4}{\der x^4}\sin x &= \sin x\\
  \frac{\der^5}{\der x^5}\sin x &= \cos x
\end{align*}
We can see that there will be a cycle of $\sin$, $\cos$, $-\sin$, and $-\cos$, repeating
indefinitely. Evaluating these derivatives at $x=0$, we have
0, 1, 0, $-1$, \ldots. All the even-order terms of the series are zero, and all the odd-order
terms are $\pm1/n!$. The result is
\begin{equation*}
  \sin x = x - \frac{1}{3!}x^3 + \frac{1}{5!}x^5 - \ldots \qquad .
\end{equation*}
The linear term is the familiar small-angle approximation $\sin x\approx x$.

The radius of convergence of this series turns out to be infinite. Intuitively the reason for
this is that the factorials grow extremely rapidly, so that the successive terms in the series eventually start
diminish quickly, even for large values of $x$.
\end{eg}

A function's Taylor series doesn't have to be evaluated around $x=0$. The Taylor series around some other
center $x=c$ is given by
\begin{equation*}
  T_c(x) = \sum_{n=0}^\infty a_n (x-c)^n \qquad ,
\end{equation*}
where 
\begin{equation*}
  \left. \frac{a_n}{n!} = \frac{\der^n y}{\der x^n}\right|_{x=c} \qquad .
\end{equation*}
To see that this is the right generalization, we can do a change of variable,
defining a new function $g(x)=f(x-c)$. The radius of convergence is to be measured
from the center $c$ rather than from 0.

\begin{eg}
\egquestion
Find the Taylor series of $\ln x$, evaluated around $x=1$.

\eganswer
Evaluating a few derivatives, we get
\begin{align*}
  \frac{\der}{\der x}\ln x &= x^{-1} \\
  \frac{\der^2}{\der x^2}\ln x &= -x^{-2}\\
  \frac{\der^3}{\der x^3}\ln x &= 2x^{-3}\\
  \frac{\der^4}{\der x^4}\ln x &= -6x^{-4}
\end{align*}
Note that evaluating these at $x=0$ wouldn't have worked, since division by zero is undefined;
this is because $\ln x$ blows up to negative infinity at $x=0$. Evaluating them at $x=1$, we
find that the $nth$ derivative equals $\pm (n-1)!$, so the coefficients of the Taylor
series are $\pm (n-1)!/n!=\pm1/n$, except for the $n=0$ term, which is zero because
$\ln 1=0$. The resulting series is
\begin{equation*}
  \ln x = (x-1) - \frac{1}{2}(x-1)^2 + \frac{1}{3}(x-1)^3 + \ldots \qquad .
\end{equation*}
We can predict that its radius of convergence can't be any greater than 1, because
$\ln x$ blows up at 0, which is at a distance of 1 from 1.
\end{eg}

\begin{hwsection}
\begin{hwwithsoln}{sequence-weierstrass}
Modify the Weierstrass definition of the limit to apply to infinite sequences.
\end{hwwithsoln}

\begin{hwwithsoln}{bogus-series}
(a) Prove that the infinite series $1-1+1-1+1-1+\ldots$ does not converge to any limit, using the generalization
of the Weierstrass limit found in problem \ref{hw:sequence-weierstrass}.
(b) Criticize the following argument. The series given in part a equals zero, because addition is associative, so we can
rewrite it as $(1-1)+(1-1)+(1-1)+\ldots$
\end{hwwithsoln}

\begin{hwwithsoln}{geometric-series-conv-integral}
Use the integral test to prove the convergence of the geometric series for $0<x<1$.
\end{hwwithsoln}

\begin{hwwithsoln}{determine-convergence}
Determine the convergence or divergence of the following series.\\
(a) $1+1/2^2+1/3^2+\ldots$\\
(b) $1/\ln\ln 3-1/\ln\ln 6+1/\ln \ln 9-1/\ln\ln 12+\ldots$\\
(c)
\begin{equation*}
\frac{2\sqrt{2}}{9801} \sum^\infty_{k=0} \frac{(4k)!(1103+26390k)}{(k!)^4 396^{4k}}
\end{equation*}
\end{hwwithsoln}

\begin{hw}
Find the Taylor series expansion of $\cos x$ around $x=0$. Check your work by combining the first two
terms of this series with the first term of the sine function from example \ref{eg:taylor-sin}
on page \pageref{eg:taylor-sin} to verify that the trig identity $\sin^2 x+\cos^2 x=1$ holds
for terms up to order $x^2$.
\end{hw}

\begin{hw}
In classical physics, the kinetic energy $K$ of an object of mass $m$ moving at velocity $v$
is given by $K=\frac{1}{2}mv^2$. For example, if a car is to start from a stoplight and then
accelerate up to $v$, this is
the theoretical minimum amount of energy  that would have to be used up by burning gasoline.
(In reality, a car's engine is not 100\% efficient, so the amount of gas burned is greater.)

Einstein's theory of relativity states that the correct equation is actually
\begin{equation*}
  K = \left(\frac{1}{\sqrt{1-\frac{v^2}{c^2}}}-1\right)mc^2 \qquad ,
\end{equation*}
where $c$ is the speed of light. The fact that it diverges as $v \rightarrow c$ is interpreted
to mean that no object can be accelerated to the speed of light.

Expand $K$ in a Taylor series, and show that the first
nonvanishing term is equal to the classical expression. This means that for velocities that
are small compared to the speed of light, the classical expression is a good approximation,
and Einstein's theory does not contradict any of the prior empirical evidence from which
the classical expression was inferred.
\end{hw}

\pagebreak

\begin{hw}
Expand $(1+x)^{1/3}$ in a Taylor series around $x=0$. The value $x=28$ lies outside
this series' radius of convergence, but we can nevertheless use it to
extract the cube root of 28 by recognizing that $28^{1/3}=3(28/27)^{1/3}$.
Calculate the root to four significant figures of precision, and check it
in the obvious way.
\end{hw}

\begin{hw}
Find the Taylor series expansion of $\log_2 x$ around $x=1$, and use it to evaluate
$\log_2 1.0595$ to four significant figures of precision. Check your result by using the
fact that 1.0595 is approximately the twelfth root of 2. This number is the ratio of
the frequencies of two successive notes of the chromatic scale\index{chromatic scale}
in music, e.g., C and D-flat.
\end{hw}

\begin{hw}\label{hw:air-res-taylor}
In free fall, the acceleration will not be exactly constant, due to air resistance. For example,
a skydiver does not speed up indefinitely until opening her chute, but rather approaches a certain
maximum velocity at which the upward force of air resistance cancels out the force of gravity.
If an object is dropped from a height $h$, and the time it takes to reach the ground is used to
measure the acceleration of gravity, $g$, then the relative error in the result due to air
resistance is\footnote{Jan Benacka
and Igor Stubna, \emph{The Physics Teacher}, 43 (2005) 432.}
\begin{align*}
  E &= \frac{g-g_{vacuum}}{g} \\
   & = 1-\frac{2b}{\ln^2\left(e^b+\sqrt{e^{2b}-1}\right)} \qquad ,
\end{align*}
where $b=h/A$, and
$A$ is a constant that depends on the size, shape, and mass of the object, and the density of
the air.   (For a sphere of mass $m$ and diameter $d$ dropping in air, $A=4.11m/d^2$. Cf. problem \ref{hw:air-res-v}, p. \pageref{hw:air-res-v}.)
Evaluate the constant and linear terms of the Taylor series for the function $E(b)$.
\end{hw}
% ans: E(b)=b/3+...

\begin{hw}
(a) Prove that the convergence of an infinite series is unaffected by omitting
some initial terms. (b) Similarly, prove that convergence is unaffected by
multiplying all the terms by some constant factor.
\end{hw}

\begin{hw}[2]
Prove the identity
\begin{equation*}
  \int_0^1 x^{-x} \der x = \sum_{n=1}^\infty n^{-n} \qquad .
\end{equation*}
This is known as the ``Sophomore's dream,'' because at first glance it looks like the
kind of plausible but false statement that someone would naively dream up.\index{Sophomore's dream}
\end{hw}

\begin{hwwithsoln}[2]{sum-of-iterated-sine}
Does $\sin x+\sin \sin x+\sin\sin\sin x+\ldots$ converge?
\end{hwwithsoln}


\end{hwsection}
